%%% Основные сведения %%%
\newcommand{\thesisAuthorLastName}{Гусев}
\newcommand{\thesisAuthorOtherNames}{Владислав Евгеньевич}
\newcommand{\thesisAuthorInitials}{В.\,Е.}
\newcommand{\thesisAuthor}             % Диссертация, ФИО автора
{%
    \texorpdfstring{% \texorpdfstring takes two arguments and uses the first for (La)TeX and the second for pdf
        \thesisAuthorLastName~\thesisAuthorOtherNames% так будет отображаться на титульном листе или в тексте, где будет использоваться переменная
    }{%
        \thesisAuthorLastName, \thesisAuthorOtherNames% эта запись для свойств pdf-файла. В таком виде, если pdf будет обработан программами для сбора библиографических сведений, будет правильно представлена фамилия.
    }
}
\newcommand{\thesisAuthorShort}        % Диссертация, ФИО автора инициалами
{\thesisAuthorInitials~\thesisAuthorLastName}
\newcommand{\thesisUdk}                % Диссертация, УДК
{\todo{000.000}}
\newcommand{\thesisTitle}              % Диссертация, название
{Каскадные схемы для обогащения регенерированного урана при его многократном рецикле в топливных циклах перспективных энергетических реакторов}
\newcommand{\thesisSpecialtyNumber}    % Диссертация, специальность, номер
{01.04.14}
\newcommand{\thesisSpecialtyTitle}     % Диссертация, специальность, название (название взято с сайта ВАК для примера)
{Теплофизика и теоретическая теплотехника}
%% \newcommand{\thesisSpecialtyTwoNumber} % Диссертация, вторая специальность, номер
%% {\todo{XX.XX.XX}}
%% \newcommand{\thesisSpecialtyTwoTitle}  % Диссертация, вторая специальность, название
%% {\todo{Теория и~методика физического воспитания, спортивной тренировки,
%% оздоровительной и~адаптивной физической культуры}}
\newcommand{\thesisDegree}             % Диссертация, ученая степень
{кандидата технических наук}
\newcommand{\thesisDegreeShort}        % Диссертация, ученая степень, краткая запись
{канд. техн. наук}
\newcommand{\thesisCity}               % Диссертация, город написания диссертации
{Москва}
\newcommand{\thesisYear}               % Диссертация, год написания диссертации
{2024}
\newcommand{\thesisOrganization}       % Диссертация, организация
{МИНИСТЕРСТВО НАУКИ И ВЫСШЕГО ОБРАЗОВАНИЯ РОССИЙСКОЙ ФЕДЕРАЦИИ\\
Федеральное государственное автономное образовательное учреждение высшего
образования\\
\textbf {<<Национальный исследовательский ядерный университет <<МИФИ>>\\(НИЯУ МИФИ)}}
\newcommand{\thesisOrganizationShort}  % Диссертация, краткое название организации для доклада
{\todo{НазУчДисРаб}}

\newcommand{\thesisInOrganization}     % Диссертация, организация в предложном падеже: Работа выполнена в ...
{Национальном исследовательском ядерном университете «МИФИ» (НИЯУ МИФИ)}

%% \newcommand{\supervisorDead}{}           % Рисовать рамку вокруг фамилии
\newcommand{\supervisorFio}              % Научный руководитель, ФИО
{Сулаберидзе Георгий Анатольевич}
\newcommand{\supervisorRegalia}          % Научный руководитель, регалии
{кандидат физико-математических наук, доцент}
\newcommand{\supervisorFioShort}         % Научный руководитель, ФИО
{Г.\,А.~Сулаберидзе}
\newcommand{\supervisorRegaliaShort}     % Научный руководитель, регалии
{к.ф.-м.н, доцент}


\newcommand{\consultOneFio}           % Второй научный руководитель, ФИО
{Смирнов Андрей Юрьевич}
\newcommand{\consultOneRegalia}       % Второй научный руководитель, регалии
{кандидат физико-математических наук}
\newcommand{\consultOneFioShort}      % Второй научный руководитель, ФИО
{А.\,Ю.~Смирнов}
\newcommand{\consultOneRegaliaShort}  % Второй научный руководитель, регалии
{к.ф.-м.н}

\newcommand{\consultTwoFio}           % Второй научный руководитель, ФИО
{Невиница Владимир Анатольевич}
\newcommand{\consultTwoRegalia}       % Второй научный руководитель, регалии
{кандидат технических наук}
\newcommand{\consultTwoFioShort}      % Второй научный руководитель, ФИО
{В.\,А.~Невиница}
\newcommand{\consultTwoRegaliaShort}  % Второй научный руководитель, регалии
{к.т.н}

\newcommand{\opponentOneFio}           % Оппонент 1, ФИО
{Палкин Валерий Анатольевич}
\newcommand{\opponentOneRegalia}       % Оппонент 1, регалии
{доктор технических наук}
\newcommand{\opponentOneJobPlace}      % Оппонент 1, место работы
{УРФУ}
\newcommand{\opponentOneJobPost}       % Оппонент 1, должность
{профессор}

\newcommand{\opponentTwoFio}           % Оппонент 2, ФИО
{Алексей Алексеевич Орлов}
\newcommand{\opponentTwoRegalia}       % Оппонент 2, регалии
{доктор технических наук}
\newcommand{\opponentTwoJobPlace}      % Оппонент 2, место работы
{Отделение ядерно-топливного цикла ТПУ}
\newcommand{\opponentTwoJobPost}       % Оппонент 2, должность
{профессор}

\newcommand{\opponentThreeFio}         % Оппонент 3, ФИО
{\todo{Фамилия Имя Отчество}}
\newcommand{\opponentThreeRegalia}     % Оппонент 3, регалии
{\todo{кандидат физико-математических наук}}
\newcommand{\opponentThreeJobPlace}    % Оппонент 3, место работы
{\todo{Основное место работы}}
\newcommand{\opponentThreeJobPost}     % Оппонент 3, должность
{\todo{старший научный сотрудник}}

\newcommand{\leadingOrganizationTitle} % Ведущая организация, дополнительные строки. Удалить, чтобы не отображать в автореферате
{\todo{Федеральное государственное бюджетное образовательное учреждение высшего
 образования -.-.-.-.-.-.-.-.-.-.-.--.-.-.-.-.-.-.-.-.-.-.-.--..--..-}}

\newcommand{\defenseDate}              % Защита, дата
{\todo{--- --- 2024~г.~в~--- часов}}
\newcommand{\defenseCouncilNumber}     % Защита, номер диссертационного совета
{{МИФИ.2.02}}
\newcommand{\defenseCouncilTitle}      % Защита, учреждение диссертационного совета
{{НИЯУ МИФИ}}
\newcommand{\defenseCouncilAddress}    % Защита, адрес учреждение диссертационного совета
{{Москва, Каширское шоссе, 31}}
\newcommand{\defenseCouncilPhone}      % Телефон для справок
{\todo{+7~(000)~00-00-00}}

\newcommand{\defenseSecretaryFio}      % Секретарь диссертационного совета, ФИО
{{Куликов Е.Г.}}
\newcommand{\defenseSecretaryRegalia}  % Секретарь диссертационного совета, регалии
{{к.т.н.}}            % Для сокращений есть ГОСТы, например: ГОСТ Р 7.0.12-2011 + http://base.garant.ru/179724/#block_30000

\newcommand{\synopsisLibrary}          % Автореферат, название библиотеки
{\todo{Название библиотеки}}
\newcommand{\synopsisDate}             % Автореферат, дата рассылки
{\todo{DD mmmmmmmm 2024 года}}

% To avoid conflict with beamer class use \providecommand
\providecommand{\keywords}%            % Ключевые слова для метаданных PDF диссертации и автореферата
{}
