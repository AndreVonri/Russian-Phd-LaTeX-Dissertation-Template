{\actuality}

Построение ядерной энергетики нового типа, устойчивой к ресурсным ограничениям и предусматривающей решение проблемы обращения с радиоактивными отходами, связано с реакторами на быстрых нейтронах, нацеленными на воспроизводство делящегося материала -- энергетического  плутония. Однако, по оценкам \cite{andrianovaPerspektivnyeToplivnyeZagruzki2015}, в ближайшие десятилетия переходного периода к двухкомпонентной ядерно-энергетической системе, делящиеся материалы будут повторно использованы в топливном цикле реакторов на тепловых нейтронах, так как они составляют основную часть парка энергоблоков.

На сегодняшний день мире в состоянии эксплуатации и сооружения насчитывают около 500 ядерных энергоблоков, подавляющее большинство из которых представляют собой легководные реакторы не тепловых нейтронах \cite{PRISHome}, работающие на ядерном топливе из низкообогащенного урана (НОУ). Работа каждого из энергоблоков создает необходимость в обеспечении его топливными ресурсами и выборе способа обращения с выгруженным из него облученным ядерным топливом (ОЯТ).

Как известно, основным материалом для производства топлива реакторов на тепловых нейтронах является природный уран, который предварительно обогащают на разделительных производствах, как правило, с использованием газовых центрифуг. В соответствии с доступными данными, мировые запасы урана  оценивают в 59 мегатонн \cite{/content/publication/d82388ab-en}. Но большую часть этих запасов, составляют ресурсы урана, для которых на сегодняшний день отсутствуют отработанные технологии добычи и, соответственно, стоимость такого урана не определена. По текущим оценкам лишь $\approx$8 мегатонн природного урана можно добыть с операционными затратами, не превышающими порогового значения в 260 \$/кг \cite{WorldDistributionUranium2018}, которое оценивается как предельное для сохранения конкурентоспособности ядерной генерации по сравнению с другими источниками (на этом уровне затраты на топливную составляющую при генерации одного кВт*ч превысят 1 евроцент (по данным 2015 года)) \cite{hartardCompetitionConflictsResource2015}. По существующим прогнозам, предсказывающие неизбежен разрыв между добычей и потреблением природного урана в будущем, что может привести к значительным проблемам  с обеспечением топливом реакторов на тепловых нейтронах в перспективе 15--20 лет \cite{international2019iaea}.

Дополнительным, а, возможно, и главным вызовом для ядерной промышленности является обращение с ОЯТ в долгосрочной перспективе. На текущий момент общемировая масса хранимого ОЯТ составляет более 400 килотонн \cite{kaygorodcevProblemyPerspektivyRazvitiya2021}. С каждым годом эта масса прирастает примерно на 11 килотонн \cite{UseReprocessedUranium2019}. Учитывая сложности поиска новых мест для строительства хранилищ ОЯТ, а также негативное отношение общества к этой проблеме во многих странах, очевидно, что без выработки приемлемого со всех точек зрения решения, это затруднит развитие ядерной энергетики в будущем. 
Следует отметить, что основным материалом облученного ядерного топлива является уран, составляющий $\approx$90-95\% его массы, за вычетом конструкционных материалов. Оставшаяся часть делится между плутонием и продуктами (осколками) деления. В большинстве случаев регенерированный уран содержит изотоп $^{235}$U с концентрацией на уровне $\geq$0,85\% (мас. доли), то есть долю делящегося изотопа выше, чем в природном уране, что делает целесообразным его повторное использование и обогащение на изотопно-разделительном производстве  \cite{NikipelovNikipelovSudby}. Вовлечение регенерированного урана отдельно или совместно с плутонием в производство ядерного топлива реакторов на тепловых нейтронах может позволить существенно сократить объем захоронения радиоактивных отходов и снизить потребности в природном уране. Однако стоит отметить, что реакторы на тепловых нейтронах являются реакторами-<<сжигателями>>, то есть в среднем воспроизводят делящихся материалов значительно меньше, чем распадается в активной зоне реактора в процессе облучения топлива. Этот факт говорит о том, что для реакторов данного типа невозможно полное замыкание ядерного топливного цикла, поскольку для их полноценного обеспечения топливом потребуются внешние источники делящихся материалов.

Следует также отметить существование проблем, связанных с переработкой ОЯТ и использованием регенерата урана в топливном цикле легководных реакторов на тепловых нейтронах.

Во-первых, переработка ОЯТ сама по себе представляет технологически сложную, капиталоемкую, радиационно-опасную и затратную процедуру. Лишь немногие страны на текущий момент обладают промышленными технологиями переработки и соответствующими мощностями, позволяющими рассматривать возможность замыкания топливного цикла.

Во-вторых, при облучении ядерного топлива в активной зоне реактора образуются искусственные изотопы урана, в первую очередь, $^{232}$U и $^{236}$U . Кроме того, как правило, возрастает и концентрация природного изотопа $^{234}$U. Изотоп $^{232}$U опасен тем, что является родоначальником цепочки распадов, среди дочерних продуктов которых есть,  в частности, $^{208}$Tl, представляющий собой источник жесткого гамма-излучения, обуславливающего высокий уровень радиоактивного фона. Поэтому при производстве уранового топлива существуют нормативные ограничения на допустимое содержание $^{232}$U в низкообогащенном уране. На текущий момент в РФ допустимые концентрации (в мас. долях) $^{232}$U в НОУ не должны превышать предельно допустимых значений, в качестве которых в Российских нормативных документах приняты следующие величины: $2\cdot10^{-7}$\% и $5\cdot10^{-7}$\%. Проблема, связанная с изотопом $^{236}$U состоит в том, что он является паразитным поглотителем нейтронов в ядерном топливе и, следовательно, отрицательно воздействует на реактивность реактора и глубину выгорания топлива. При наличии в загружаемом в реактор топливе $^{236}$U для компенсации его отрицательного влияния и получения заданных ядерно-физических характеристик реактора нужно повышать среднее начальное обогащение топлива по $^{235}$U.  Отдельно стоит подчеркнуть, что концентрации изотопов $^{232}$U, $^{234}$U и $^{236}$U (четных изотопов), возрастают при обогащении регенерированного урана в ординарных (трехпоточных) каскадах газовых центрифуг, используемых для обогащения природного урана. Под <<ординарным>> каскадом понимают каскад, имеющий три внешних потока -- питание, отбор и отвал. Фактически это означает необходимость развития способов обогащения регенерированного урана с учетом требований к изотопному составу производимого обогащенного продукта, отвечающих действующим техническим условиям на товарный низкообогащенный уран.

Настоящая диссертационная работа посвящена теоретическому обоснованию эффективных способов решения второй проблемы -- обогащение регенерированного урана. Что касается переработки ОЯТ, предваряющей стадию обогащения восстановленного посредством нее регенерата, то в данной работе не будут затронуты теоретические и практические вопросы переработки ОЯТ. При этом отметим, что на текущий момент в РФ и некоторых странах существуют отработанные технологии переработки ОЯТ, что дает основания рассматривать и анализировать вопросы эффективного обогащения регенерированного урана, считая, что проблема его выделения из ОЯТ решена \cite{KoncepciyaPoObrashcheniyu,gaoEconomicPotentialFuel2017,moratillasoriaRecyclingLongTermStorage2013,bouchardECONOMICSNUCLEARENERGY}.

Перейдем к анализу проблем, возникающих для технологий разделения изотопов в контексте обогащения регенерированного урана. На сегодняшний день предложен ряд технических решений, позволяющих решить задачу обогащения регенерированного урана до концентраций $^{235}$U , требуемых в современных топливных циклах энергетических реакторов на тепловых нейтронах (в частности отечественных ВВЭР), при одновременном выполнении принятых ограничений на содержание $^{232}$U в ядерном топливе и реализации необходимого дообогащения регенерата по $^{235}$U для компенсации негативного влияния $^{236}$U. Тем не менее далеко не все из них способны решить задачу обогащения регенерата с одновременной коррекцией его изотопного состава в условиях, когда исходное содержание четных изотопов может существенно меняться, например, в сторону увеличения. Последнее обстоятельство особо важно в контексте рассмотрения перспективных реакторов, имеющих относительно высокую глубину выгорания топлива и, как следствие, состав ОЯТ которых может характеризоваться относительно высоким содержанием четных изотопов. Помимо этого, необходимо учитывать, что замыкание топливного цикла реакторов на тепловых нейтронах подразумевает многократное обращение урана в топливе, что будет обуславливать дополнительное накопление четных изотопов в регенерате от цикла к циклу, учитывая, что при таком подходе исходное топливо на каждом цикле будет содержать четные изотопы еще до загрузки в реактор.

Очевидно, что вопросы коррекции изотопного состава регенерированного урана лежат в области теории и практики разделения изотопных смесей. Данное обстоятельство делает актуальной для разделительной науки задачу поиска эффективных способов обогащения регенерата урана с одновременной коррекцией его изотопного состава в условиях развития тенденции повышения глубины выгорания топлива и многократного использования урана в нем (многократный рецикл урана). В дополнение к этому отдельно стоит вопрос выбора оптимальной каскадной схемы для дообогащения регенерированного урана, которая должна обеспечить максимально эффективное использование ресурса регенерированного урана при минимальных затратах работы разделения.

Для теоретического обоснования возможных способов решения указанных задач целесообразно использовать активно развивающуюся в последние десятилетия теорию каскадов для разделения многокомпонентных изотопных смесей. Существующие в этой теории модели массопереноса компонентов в многоступенчатых разделительных установках позволяют анализировать ключевые закономерности изменения интегральных характеристик таких установок в процессе разделения изотопных смесей, включая и смесь регенерированного урана, с целью поиска оптимальных условиях такого процесса.
 

{\aim} диссертационной работы является теоретическое обоснование эффективных способов обогащения регенерированного урана в каскадах центрифуг при его многократном использовании в регенерированном ядерном топливе для реакторов на тепловых нейтронах.

Для~достижения поставленной цели решены следующие {\tasks}:
\begin{enumerate}
  \item Проанализированы предложенные к текущему моменту способы обогащения регенерированного урана с учетом ограничений на концентрации четных изотопов в товарном НОУ. Теоретически оценена возможность их применения для решения задачи оптимального использования поступающей в обогащение массы регенерированного урана. 
  \item Выявлены физические ограничения решения задачи обогащения регенерата произвольного изотопного состава в одиночном каскаде и простых модификациях двойного каскада при одновременном выполнении условий на концентрации изотопов $^{232}$U, $^{234}$U и $^{236}$U в получаемом продукте -- низкообогащенном уране, и расходовании заданной массы регенерата на единицу получаемого продукта.
  \item Физически обоснованы принципы построения двойных каскадов,
  позволяющих корректировать изотопный состав регенерата по концентрациям
  изотопов $^{232}$U, $^{234}$U и $^{236}$U с одновременным расходованием максимального количества
  подлежащего обогащению регенерата при различных исходных концентрациях
  четных изотопов в нем. Разработаны и апробированы оригинальные методики расчета и оптимизации предложенной модификации двойного каскада. Показана возможность использования предложенной схемы при различных внешних условиях, а также различных концентрациях четных изотопов в исходном регенерированном уране.
  \item Обоснован способ эффективной «утилизации» загрязненной четными изотопами фракции, возникающей в двойных каскадах при очистке от $^{232}$U, с учетом полной или
  частичной подачи данной фракции: а) в третий каскад с предварительным
  перемешиванием ее с природным, обедненным и/или низкообогащенным ураном; б) в отдельный двойной каскад, осуществляющий наработку низкообогащенного урана для последующей топливной кампании реактора.
  \item Изучены физические закономерности изменения изотопного состава регенерата урана в зависимости от выбора параметров модифицированного двойного каскада при обогащении регенерированного урана с различным исходным содержанием четных изотопов в питающей смеси.
  % \item Обобщение и систематизация подходов к выбору каскадной схемы, позволяющих
  % эффективное обогащение регенерированного урана в условиях однократного и
  % многократного рецикла.
  % \item Определение физических закономерностей изменения изотопного состава
  % регенерированного урана и параметров модифицированного двойного каскада для
  % его дообогащения при многократном рецикле урана (отдельно и совместно с
  % плутонием) в топливе реакторов типа ВВЭР.
\end{enumerate}


{\novelty}
\begin{enumerate}
  \item Впервые предложены оригинальные методики расчета различных модификаций двойных каскадов, позволяющие корректировать
  изотопный состав регенерата по концентрациям изотопов $^{232}$U, $^{234}$U и $^{236}$U с одновременным расходованием всего подлежащего обогащению регенерата при различных исходных концентрациях четных изотопов в нем и различных внешних условиях.
  \item Обоснованы физические принципы построения тройных и двойных каскадных схем для <<утилизации>> фракции, загрязненной четными изотопами при обогащении регенерированного урана в двойных каскадных схемах.
  \item Выполнены оригинальные исследования по изучению физических закономерностей изменения изотопного состава регенерата
  от конфигураций каскадных схем (под конфигурацией каскада понимаем его технологические параметры: количества ступеней в обеднительной и обогатительной части, номера ступеней с подачей питающих смесей, а также функцию распределения потока по длине каскада) и интегральных характеристик модифицированных двойных и тройных каскадах при обогащении регенерированного урана с различным исходным содержанием четных изотопов.
  % \begin{enumerate}
  %   \item расход природного урана в цикле;
  %   \item затраты работы разделения в цикле;
  %   \item доля потерь $^{235}$U в каскадной схеме;
  %   \item доля потерь $^{235}$U из исходного регенерата;
  %   \item доля газовых центрифуг в схеме, в которых превышена предельно допустимая концентрация по $^{232}$U.
  % \end{enumerate}
  \item Проанализированы различные подходы к утилизации высокоактивного «нештатного» отхода, образующегося в процессе обогащения регенерированного урана в двойном каскаде.

\end{enumerate}

{\influence} 
\begin{enumerate}
  \item Разработаны модификации двойных и тройных каскадов, позволяющие обогащать регенерированный уран с одновременным выполнением ограничений на концентрации четных изотопов и вовлечением требуемой массы регенерата.
  % \item Проведенный анализ результатов расчетного моделирования молекулярно-селективного массопереноса в модифицированных двойных и тройных каскадах для обогащения регенерата урана выявляет область практической применимости подобных схем для получения НОУ-продукта на основе регенерированного урана.
  \item Предложенные методики оптимизации параметров каскадных схем двойного и тройного каскадов позволяют находить наиболее эффективные с точки зрения таких критериев, как расход работы разделения, расход природного урана, степень извлечения $^{235}$U, наборы их параметров, позволяющие осуществлять возврат всей массы регенерированного урана в цикл и одновременно удовлетворять ограничения по концентрациям четных изотопов. Предложенные методики оптимизации систем каскадов могут быть адаптированы к расчету и оптимизации параметров различных вариантов каскадных схем для многокомпонентных смесей неурановых элементов.
  \item Полученные результаты могут быть использованы в расчетных группах на предприятиях и организациях, связанных как с проектированием и построением разделительных каскадов, так и непосредственным производством изотопной продукции (АО «Уральский электрохимический комбинат», АО «Сибирский химический комбинат», АО «ТВЭЛ», АО «Восточно-Европейский головной научно-исследовательский и проектный институт энергетических технологий», АО «ПО «ЭХЗ» и др.). 
  \item Предложенные методики расчета могут лечь в основу имитационных моделей топливного цикла реакторов на тепловых нейтронах, использующих регенерированное урановое топливо.  
  % \item Разработан тренировочный программный комплекс для расчета каскада, нацеленного на возврат регенерированного урана. Код оформлен в виде лабораторной работы, которая внедрена в учебный процесс.
\end{enumerate}


{\methods}.
Исследование проводит систематизацию научно-технической литературы, посвященной заявленной теме.
Применены подходы, известные в современной теоретической физике, и в частности, в теории разделения изотопов в каскадах.
В работе теоретически обоснованы принципы построения анализируемых каскадов, разработаны программные коды расчета и оптимизации их параметров для различных постановок задач, проведено их компьютерное моделирование.
При разработке программных кодов использована теория модельных каскадов -- квазиидеального каскада и его разновидности R-каскада, для которого выполняется условие несмешивания относительных концентраций пары выбранных компонентов. При подготовке программных кодов использованы современные программные средства языков программирования Julia и Python и подключаемых библиотек, таких как NLsolve.jl, Optim.jl, SciPy, предназначенных для решения систем нелинейных уравнений и оптимизационных процедур, Matplotlib и PGFPlots.jl для визуализации результатов.


{\defpositions}
\begin{enumerate}
  \item Физико-математические модели, методики расчета и оптимизации модифицированных двойных и тройных каскадных схем для обогащения регенерата урана с одновременным выполнением условий на концентрации четных изотопов и максимальным вовлечением исходного материала.
  \item Результаты анализа физических закономерностей массопереноса компонентов смеси регенерированного урана в ординарном каскаде, позволяющие однозначно определить условия при которых возможно/невозможно получение необходимого количества конечного продукта на основе регенерированного урана различного исходного состава путем обогащения в одиночном каскаде.
  \item Практические рекомендации по <<утилизации>> загрязненной четными изотопами фракции, получаемой в двойных каскадах при обогащении регенерата. 
\end{enumerate}

{\reliability}.
Надежность, достоверность и обоснованность научных положений и выводов, сделанных в диссертации, следует из корректности постановки задач, физической обоснованности применяемых приближений, использования методов, ранее примененных в аналогичных исследованиях, взаимной согласованности результатов. Корректность результатов вычислительных экспериментов гарантируется тестами и операторами проверки соответствия ограничениям, верифицирующими строгое выполнение заданных условий и соблюдение условий сходимости балансов (массовых и покомпонентных).

% {\probation}
% См. приложение А2.

{\contribution} Автор принимал участие разработке каскадных схем, написании программных кодов, проведении вычислительных экспериментов, а также в обработке и в анализе результатов численных экспериментов.

% {\publications} 
% См. приложение А1.

