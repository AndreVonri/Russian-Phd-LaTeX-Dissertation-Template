{\actuality}

Подавляющее большинство из $\approx$500 эксплуатируемых и сооружаемых ядерных энергоблоков представляют собой легководные реакторы на тепловых нейтронах \cite{PRISHome}, работающие на ядерном топливе из низкообогащенного урана (НОУ). Работа каждого из энергоблоков создает необходимость в обеспечении его топливом и выборе способа обращения с выгруженным из него облученным ядерным топливом (ОЯТ).

Основным материалом для производства топлива реакторов на тепловых нейтронах является природный уран, обогащаемый в каскадах газовых центрифуг. Но основную часть его мировых запасов можно добыть только при высоких операционных затратах, которые оцениваются как неконкурентоспособные для ядерной генерации энергии по сравнению с другими источниками \cite{Uranium2022,WorldDistributionUranium2018,hartardCompetitionConflictsResource2015}. 

Еще одним вызовом для ядерной промышленности является обращение с ОЯТ, общемировая масса которого превышает 400 килотонн и прирастает ежегодно на $>$15 килотонн \cite{kaygorodcevProblemyPerspektivyRazvitiya2021,UseReprocessedUranium2020WNA}. При этом, основным материалом облученного ядерного топлива является уран, составляющий $\approx$90-95\% его массы, за вычетом конструкционных материалов, концентрация изотопа $^{235}$U в котором, как правило, выше, чем в природном уране, что делает целесообразным его повторное использование \cite{NikipelovNikipelovSudby}. Его вовлечение в производство ядерного топлива реакторов на тепловых нейтронах может позволить существенно сократить объем захоронения радиоактивных отходов и снизить потребности в природном уране.
% Однако стоит отметить, что реакторы на тепловых нейтронах являются реакторами-<<сжигателями>>, то есть в среднем воспроизводят делящихся материалов значительно меньше, чем распадается в активной зоне реактора в процессе облучения топлива. Этот факт говорит о том, что для реакторов данного типа невозможно полное замыкание ядерного топливного цикла, поскольку для их полноценного обеспечения топливом потребуются внешние источники делящихся материалов, а также обогащение исходной урановой смеси до требуемого содержания $^{235}$U, характерного для топлива легководного реактора ($\approx$5\%). 

Вовлечение регенерата сопряжено с рядом проблем, так как при облучении ядерного топлива в активной зоне реактора образуются искусственные изотопы урана, в первую очередь, $^{232}$U и $^{236}$U. Кроме того, как правило, возрастает и концентрация природного изотопа $^{234}$U. Изотоп $^{232}$U опасен тем, что является родоначальником цепочки распадов, среди дочерних продуктов которых есть,  в частности, $^{208}$Tl, представляющий собой источник жесткого гамма-излучения, обуславливающего высокий уровень радиоактивного фона. Поэтому при производстве уранового топлива существуют нормативные ограничения на допустимое содержание $^{232}$U в низкообогащенном уране. На текущий момент в РФ допустимые концентрации (в мас. долях) $^{232}$U в НОУ не должны превышать предельно допустимых значений: $2\cdot10^{-7}$\% или $5\cdot10^{-7}$\%. Проблема, связанная с изотопом $^{236}$U, состоит в том, что он является паразитным поглотителем нейтронов в ядерном топливе и, следовательно, отрицательно воздействует на реактивность реактора и глубину выгорания топлива. Для компенсации отрицательного влияния $^{236}$U и получения заданных ядерно-физических характеристик реактора нужно повышать среднее начальное обогащение топлива по $^{235}$U.  При этом, концентрации изотопов $^{232}$U, $^{234}$U и $^{236}$U (четных изотопов), возрастают при обогащении регенерированного урана в ординарных (трехпоточных) каскадах газовых центрифуг, используемых для обогащения природного урана. Фактически это означает необходимость развития способов обогащения регенерированного урана с учетом требований к изотопному составу производимого обогащенного продукта, отвечающих действующим техническим условиям на товарный низкообогащенный уран.

Настоящая диссертационная работа посвящена разработке эффективных способов решения второй проблемы, связанной с обогащением регенерированного урана.
Перейдем к анализу проблем, возникающих в ее контексте для технологий разделения изотопов. На сегодняшний день предложен ряд технических решений, позволяющих решить задачу обогащения регенерированного урана до концентраций $^{235}$U, требуемых в современных топливных циклах энергетических реакторов на тепловых нейтронах (в частности отечественных ВВЭР), при одновременном выполнении принятых ограничений на содержание $^{232}$U в ядерном топливе и реализации необходимого дообогащения регенерата по $^{235}$U для компенсации негативного влияния $^{236}$U. Тем не менее, далеко не все из них способны решить задачу обогащения регенерата с одновременной коррекцией его изотопного состава в условиях, когда исходное содержание четных изотопов может меняться в широком диапазоне. Последнее обстоятельство особо важно в контексте рассмотрения перспективных реакторов, имеющих относительно высокую глубину выгорания топлива и, как следствие, состав ОЯТ которых может характеризоваться повышенным содержанием четных изотопов (кратно больше предельно допустимых значений). Помимо этого, необходимо учитывать, что замыкание топливного цикла реакторов на тепловых нейтронах подразумевает многократное обращение урана в топливе, что будет обуславливать дополнительное накопление четных изотопов в регенерате от цикла к циклу, учитывая, что при таком подходе исходное топливо на каждом цикле будет содержать четные изотопы еще до загрузки в реактор.

Очевидно, что вопросы коррекции изотопного состава регенерированного урана лежат в области теории и практики разделения изотопных смесей, что делает актуальной для разделительной науки задачу поиска эффективных способов обогащения регенерата урана с одновременной коррекцией его изотопного состава в условиях развития тенденции повышения глубины выгорания топлива и многократного использования урана в нем (многократный рецикл урана). В дополнение к этому важен выбор оптимальной каскадной схемы, которая должна обеспечить максимально эффективное использование ресурса регенерированного урана при минимальных затратах работы разделения.

Разработка способов решения указанных задач возможна с использованием теории каскадов для разделения многокомпонентных изотопных смесей, описывающей массоперенос компонентов в многоступенчатых разделительных установках и позволяющей находить оптимальные условия такого процесса.


{\aim} диссертационной работы является разработка способов обогащения регенерированного урана в каскадах центрифуг при его многократном использовании в регенерированном ядерном топливе для реакторов на тепловых нейтронах.

Для~достижения поставленной цели решены следующие {\tasks}:
\begin{enumerate}
  \item Выявлены физические ограничения решения задачи обогащения регенерата произвольного изотопного состава в одиночном каскаде и в простых модификациях двойного каскада при одновременном выполнении условий на концентрации изотопов $^{232}$U, $^{234}$U и $^{236}$U в получаемом продукте --- низкообогащенном уране, и расходовании заданной массы регенерата на единицу получаемого продукта.
  \item Предложены модификации двойных каскадов, позволяющие корректировать изотопный состав регенерата по концентрациям изотопов $^{232}$U, $^{234}$U и $^{236}$U с одновременным расходованием максимального количества подлежащего обогащению регенерата при различных исходных концентрациях четных изотопов в нем. Разработаны и апробированы методики расчета и оптимизации предложенной модификации двойного каскада. Показана возможность использования предложенной схемы при различных внешних условиях, а также различных концентрациях четных изотопов в исходном регенерированном уране.
  \item Обоснованы способы вовлечения загрязненной четными изотопами фракции, возникающей в двойных каскадах при очистке от $^{232}$U, с учетом полной или частичной подачи данной фракции: а) в отдельный двойной каскад, осуществляющий наработку низкообогащенного урана для последующей топливной кампании реактора; б) перемешивании этой фракции с потоками обедненного урана и низкообогащенного урана для получения дополнительной массы товарного НОУ; в) в третий каскад с предварительным перемешиванием ее с природным, обедненным и/или низкообогащенным ураном. Выявлены достоинства и недостатки каждого из способов, что позволяет обозначить возможные области их применения. Для предложенной в случае в) системы каскадов разработана методика оптимизации её параметров по различным критериям эффективности. На основе разработанной методики показаны возможность обеспечить экономию природного урана в цикле вплоть до 30\% по отношению к открытому топливному циклу.
  \item Изучены физические закономерности изменения изотопного состава регенерата урана в зависимости от выбора параметров модифицированного двойного каскада при обогащении регенерированного урана с различным исходным содержанием четных изотопов в питающей смеси.
  % \item Обобщение и систематизация подходов к выбору каскадной схемы, позволяющих эффективное обогащение регенерированного урана в условиях однократного и многократного рецикла.
  % \item Определение физических закономерностей изменения изотопного состава регенерированного урана и параметров модифицированного двойного каскада для его дообогащения при многократном рецикле урана (отдельно и совместно с плутонием) в топливе реакторов типа ВВЭР.
\end{enumerate}

{\novelty}
\begin{enumerate}
  \item Разработаны способы обогащения регенерированного урана на основе построения тройных и двойных каскадных схем для вовлечения фракции, загрязненной четными изотопами при обогащении регенерированного урана в двойных каскадных схемах.
  \item Предложены методики расчета различных модификаций двойных каскадов, позволяющие корректировать изотопный состав регенерата по концентрациям изотопов $^{232}$U, $^{234}$U и $^{236}$U с одновременным расходованием всего подлежащего обогащению регенерата при различных исходных концентрациях четных изотопов в нем и различных внешних условиях.
  \item Изучены физические закономерности изменения интегральных характеристик модифицированных двойных и тройных каскадах изотопного состава регенерата при его обогащении в таких каскадных схемах для различных внешних условий и различных составов поступившего в обогащение регенерата.
  \item Предложены подходы к вовлечению высокоактивного нештатного отхода, образующегося в процессе обогащения регенерированного урана в двойном каскадe, в воспроизводство ядерного топлива.
\end{enumerate}

{\influence} 
\begin{enumerate}
  \item Разработаны модификации двойных и тройных каскадов, позволяющие обогащать регенерированный уран с одновременным выполнением ограничений на концентрации четных изотопов и вовлечением требуемой массы регенерата.
  % \item Проведенный анализ результатов расчетного моделирования молекулярно-селективного массопереноса в модифицированных двойных и тройных каскадах для обогащения регенерата урана выявляет область практической применимости подобных схем для получения НОУ-продукта на основе регенерированного урана.
  \item Предложены методики оптимизации параметров каскадных схем двойного и тройного каскадов, позволяющие находить наиболее эффективные с точки зрения таких критериев, как расход работы разделения, расход природного урана, степень извлечения $^{235}$U, наборы их параметров, при одновременном возврате всей массы регенерированного урана в цикл и выполнении ограничений по концентрациям четных изотопов. Предложенные методики оптимизации систем каскадов могут быть адаптированы к расчету и оптимизации параметров различных вариантов каскадных схем для разделения многокомпонентных смесей неурановых элементов.
  \item Полученные результаты могут быть использованы в расчетных группах на предприятиях и организациях, связанных как с проектированием и построением разделительных каскадов, так и непосредственным производством изотопной продукции (АО «Уральский электрохимический комбинат», АО «Сибирский химический комбинат», АО «ТВЭЛ», АО «Восточно-Европейский головной научно-исследовательский и проектный институт энергетических технологий», АО «ПО «ЭХЗ» и др.).
  \item Предложенные расчетные методики могут лечь в основу имитационных моделей и цифровых двойников технологий топливного цикла реакторов на тепловых нейтронах, использующих регенерированное урановое топливо.  
\end{enumerate}

{\methods}
Исследование проводит систематизацию научно-технической литературы, посвященной заявленной теме.
Применены подходы, известные в современной теоретической физике, и в частности, в теории разделения изотопов в каскадах.
В работе теоретически обоснованы принципы построения анализируемых каскадов, разработаны программные коды расчета и оптимизации их параметров для различных постановок задач, проведено их компьютерное моделирование.
При разработке программных кодов использована теория квазиидеального каскада. При подготовке программных кодов использованы современные программные средства языков программирования Julia и Python и подключаемых библиотек, таких как NLsolve.jl, Optim.jl, SciPy, предназначенных для решения систем нелинейных уравнений и оптимизационных процедур, Matplotlib и PGFPlots.jl для визуализации результатов.

{\defpositions}
\begin{enumerate}
  \item Способы обогащения регенерата урана с одновременным выполнением условий на концентрации четных изотопов и максимальным вовлечением исходного материала в каскадных схемах. Подходы к оптимизации предложенных каскадных схем;
  \item Условия, при которых возможно/невозможно получение необходимого количества конечного продукта на основе регенерированного урана различного исходного состава путем его обогащения в таком каскаде;
  \item Обоснования возможности решения задачи обогащения регенерата в модифицированном двойном каскаде в широком диапазоне изменения внешних условий;
  \item Способы вовлечения загрязненной четными изотопами фракции, получаемой в двойных каскадах при обогащении регенерата. 
\end{enumerate}

{\reliability}.
Надежность, достоверность и обоснованность научных положений и выводов, сделанных в диссертации, следует из корректности постановки задач, физической обоснованности применяемых приближений, использования методов, ранее примененных в аналогичных исследованиях, взаимной согласованности результатов. Корректность результатов вычислительных экспериментов гарантируется тестами и операторами проверки соответствия ограничениям, верифицирующими строгое выполнение заданных условий и соблюдение условий сходимости балансов (массовых и покомпонентных).

{\probation}
Результаты, изложенные в материалах диссертации, доложены и обсуждены на конференциях:
\begin{itemize}
  \item V Международная научная конференция молодых ученых, аспирантов и студентов «Изотопы: технологии, материалы и применение», г. Томск, Россия, 2018;
  \item VI Международная научная конференция молодых ученых, аспирантов и студентов «Изотопы: технологии, материалы и применение», г. Томск, Россия, 2020;
  \item 15th International Workshop on Separation Phenomena in Liquids and Gases (SPLG-2019), г. Уси, Китай, 2019;
  \item 16th International Workshop on Separation Phenomena in Liquids and Gases (SPLG-2021), г. Москва, Россия, 2021;
  \item XVII International conference and School for young scholars “Physical chemical processes in atomic systems”, г. Москва, Россия, 2019.
\end{itemize}

По теме диссертации опубликовано 9 печатных работ, в том числе 5 в изданиях, индексированных в международной системе цитирования Scopus. Автор принимал участие в следующих проектах, поддержанных Российским научным фондом (РНФ), в которых были использованы некоторые из результатов диссертационной работы: 
\begin{itemize}
  \item Разработка каскадных схем для эффективного получения изотопно-модифицированных материалов для топливных циклов перспективных ядерных реакторов и других приложений (2018---2020 гг.);
  \item Оптимизация стационарного и нестационарного массопереноса в многокаскадных схемах для получения стабильных изотопов и обогащения регенерированного урана (2020---2022 гг.).
\end{itemize}


{\contribution} Автор принимал участие разработке каскадных схем, написании программных кодов, проведении вычислительных экспериментов, а также в обработке и анализе результатов вычислительных экспериментов.
