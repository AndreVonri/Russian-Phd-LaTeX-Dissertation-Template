
{\actuality}

Построение ядерной энергетики нового типа, устойчивой к ресурсным ограничениям и предусматривающей решение проблемы обращения с радиоактивными отходами, связано с реакторами на быстрых нейтронах, обладающими размножающими свойствами. То есть, такая система, называемая двухкомпонентной ядерной энергетикой, нацелена на воспроизводство делящегося материала -- энергетического  плутония -- в реакторе на быстрых нейтронах. Однако, по оценкам (ждем новейшие сценарные анализы КИ), в ближайшие десятилетия, по мере становления двухкомпонентной ядерно-энергетической системы, неизбежен переходный период, когда делящиеся материалы будут повторно использоваться в топливном цикле реакторов на тепловых нейтронах, так как они составляют основную часть парка энергоблоков. Основным материалом топлива является уран, составляющий $\approx$95\% за вычетом конструкционных материалов. К тому же, оценки показывают, что регенерированный уран с содержанием $^{235}$U на уровне $\approx$0,85-0,95\% экономически целесообразно дообогащать на изотопно-разделительном производстве \cite{NikipelovNikipelovSudby}.

Итак, использование выделенного из отработавшего ядерного топлива (ОЯТ) регенерированного урана является основным достижимым в ближайшей перспективе направлением вовлечения регенерируемых материалов в топливный цикл энергетических реакторов. Выделенный из ОЯТ регенерированный уран может быть использован в составе топлива ВВЭР различными способами: центрифужное дообогащение для производства уранового топлива, дообогащение и включение в состав топлива типа REMIX-В \cite{zilbermanVozmozhnostIspolzovaniyaTopliva2012}, использование в составе МОX-топлива вместо обедненного природного урана. Отметим, что такое рециклирование можно осуществлять многократно, \textcolor{red}{однако в работе \cite{postovarovaRadiacionnyeHarakteristikiRemikstopliva} делается оговорка, что REMIX-В не рециклируется, что требует проверки}.

Рецикл урана является сложной задачей ввиду присутствия в изотопном составе регенерата ряда четных изотопов. В первую очередь, это неприродные $^{232}$U и $^{236}$U. Присутствие первого затрудняет обращение с регенератом, как на стадии обогащения, так и на стадии производства твэлов. Влияние же второго сказывается на ухудшении размножающих свойств ядерного топлива, поскольку данный изотоп является паразитным поглотителем тепловых нейтронов. Вдобавок, в регенерате, по сравнению с природным ураном, на порядок выше содержание $^{234}$U При этом, ориентируясь на сегодняшние тенденции к увеличению длительности топливных циклов ВВЭР, которые связаны с повышением глубины выгорания топлива, следует принять во внимание вытекающий из этого рост содержания вредных четных изотопов в регенерате.

Из вышесказанного, ввиду необходимости решения задачи эффективного вовлечения регенерированного урана в ядерный топливный цикл (ЯТЦ), существует потребность поиска и дальнейшей разработки каскадных схем, которые позволят решить задачу производства на основе регенерата свежего топлива, удовлетворяющего стандартным спецификациям.

Работа актуальна в свете принятой ГК Росатом программы "Сбалансированный ЯТЦ", нацеленной на обеспечение дополнительных конкурентных преимуществ направления зарубежных поставок.
% {\actuality} Обзор, введение в тему, обозначение места данной работы в
% мировых исследованиях и~т.\:п., можно использовать ссылки на~другие
% работы~\autocite{Gosele1999161}
% (если их~нет, то~в~автореферате
% автоматически пропадёт раздел <<Список литературы>>). Внимание! Ссылки
% на~другие работы в~разделе общей характеристики работы можно
% использовать только при использовании \verb!biblatex! (из-за технических
% ограничений \verb!bibtex8!. Это связано с тем, что одна
% и~та~же~характеристика используются и~в~тексте диссертации, и в
% автореферате. В~последнем, согласно ГОСТ, должен присутствовать список
% работ автора по~теме диссертации, а~\verb!bibtex8! не~умеет выводить в одном
% файле два списка литературы).
% При использовании \verb!biblatex! возможно использование исключительно
% в~автореферате подстрочных ссылок
% для других работ командой \verb!\autocite!, а~также цитирование
% собственных работ командой \verb!\cite!. Для этого в~файле
% \verb!common/setup.tex! необходимо присвоить положительное значение
% счётчику \verb!\setcounter{usefootcite}{1}!.

% Для генерации содержимого титульного листа автореферата, диссертации
% и~презентации используются данные из файла \verb!common/data.tex!. Если,
% например, вы меняете название диссертации, то оно автоматически
% появится в~итоговых файлах после очередного запуска \LaTeX. Согласно
% ГОСТ 7.0.11-2011 <<5.1.1 Титульный лист является первой страницей
% диссертации, служит источником информации, необходимой для обработки и
% поиска документа>>. Наличие логотипа организации на~титульном листе
% упрощает обработку и~поиск, для этого разметите логотип вашей
% организации в папке images в~формате PDF (лучше найти его в векторном
% варианте, чтобы он хорошо смотрелся при печати) под именем
% \verb!logo.pdf!. Настроить размер изображения с логотипом можно
% в~соответствующих местах файлов \verb!title.tex!  отдельно для
% диссертации и автореферата. Если вам логотип не~нужен, то просто
% удалите файл с~логотипом.

\ifsynopsis
Этот абзац появляется только в~автореферате.
Для формирования блоков, которые будут обрабатываться только в~автореферате,
заведена проверка условия \verb!\!\verb!ifsynopsis!.
Значение условия задаётся в~основном файле документа (\verb!synopsis.tex! для
автореферата).
\else
% Этот абзац появляется только в~диссертации.
% Через проверку условия \verb!\!\verb!ifsynopsis!, задаваемого в~основном файле
% документа (\verb!dissertation.tex! для диссертации), можно сделать новую
% команду, обеспечивающую появление цитаты в~диссертации, но~не~в~автореферате.
\fi

% {\progress}
% Этот раздел должен быть отдельным структурным элементом по
% ГОСТ, но он, как правило, включается в описание актуальности
% темы. Нужен он отдельным структурынм элемементом или нет ---
% смотрите другие диссертации вашего совета, скорее всего не нужен.

{\aim} диссертационной работы является изучение физических закономерностей
молекулярно-селективного массопереноса в ординарных и многопоточных каскадах
для разделения многокомпонентных смесей с целью дальнейшего поиска
оптимальных условий обогащения регенерированного урана в подобных каскадах при
его многократном использовании в различных видах регенерированного ядерного
топлива для реакторов на тепловых нейтронах.

Для~достижения поставленной цели необходимо было решить следующие {\tasks}:
\begin{enumerate}
  \item Анализ физических закономерностей массопереноса компонентов смеси
  регенерированного урана в ординарном каскаде. Выявлены физические причины
  невозможности решения задачи обогащения регенерата произвольного изотопного
  состава в одиночном каскаде при одновременном выполнении условий на
  концентрации изотопов $^{232}$U, $^{234}$U и $^{236}$U в получаемом продукте – низкообогащенном уране.
  \item Физическое обоснование принципов построения двойных каскадов,
  позволяющих корректировать изотопный состав регенерата по концентрациям
  изотопов $^{232}$U, $^{234}$U и $^{236}$U с одновременным расходованием полного количества
  подлежащего обогащению регенерата при различных исходных концентрациях
  четных изотопов в нем.
  \item Обоснование физических принципов «утилизации» загрязненной четными
  изотопами фракции, возникающей в двойных каскадах, путем полной или
  частичной подачи данной фракции в третий каскад с предварительным
  перемешиванием ее с природным и/или низкообогащенным ураном.
  \item Изучение физических закономерностей изменения изотопного состава регенерата и
  интегральных характеристик модифицированных двойных каскадах и тройных
  каскадах при обогащении регенерированного урана с различным исходным
  содержанием четных изотопов.
  \item Обобщение и систематизация подходов к выбору каскадной схемы, позволяющих
  эффективное обогащение регенерированного урана в условиях однократного и
  многократного рецикла.
  \item Определение физических закономерностей изменения изотопного состава
  регенерированного урана и параметров модифицированного двойного каскада для
  его дообогащения при многократном рецикле урана (отдельно и совместно с
  плутонием) в топливе реакторов типа ВВЭР.
\end{enumerate}


{\novelty}
\begin{enumerate}
  \item Впервые предложены модификации двойных каскадов, позволяющих корректировать
  изотопный состав регенерата по концентрациям изотопов $^{232}$U, $^{234}$U и $^{236}$U с одновременным расходованием полного количества подлежащего обогащению регенерата при различных исходных концентрациях четных изотопов в нем.
  \item Обоснованы физические принципы построения тройных каскадных схем для максимально полного использования использования исходного регенерированного урана для воспроизводства топлива реакторов на тепловых нейтронах.
  \item Выполнены оригинальные исследования по изучению физических закономерностей изменения изотопного состава регенерата и интегральных характеристик модифицированных двойных и тройных каскадах при обогащении регенерированного урана с различным исходным содержанием четных изотопов.
  \item Разработан обобщенный подход к выбору каскадной схемы для эффективного обогащения регенерированного урана в условиях однократного и многократного рецикла.
  \item Разработан алгоритм оптимизации модифицированных двойных и тройных каскадов для обогащения регенерата урана по различным критериям
  эффективности.
  \item Определены физические закономерности изменения изотопного состава регенерированного урана и параметров модифицированного двойного и тройного каскадов для его дообогащения при многократном рецикле урана (отдельно и совместно с плутонием) в топливе реакторов типа ВВЭР.
\end{enumerate}

{\influence} 
\begin{enumerate}
  \item Проведенный анализ физических закономерностей массопереноса компонентов смеси регенерированного урана в ординарном каскаде позволил однозначно определить условия при которых возможно/невозможно обогащение регенерированного урана различного исходного состава в одиночном каскаде.
  \item Разработанные модификации двойных и тройных каскадов позволяют эффективно решать задачу обогащения регенерированного урана с одновременным выполнением ограничений на концентрации четных изотопов и максимальным использованием исходного регенерата.
  \item Анализ результатов расчетного моделирования молекулярно-селективного массопереноса в модифицированных двойных и тройных каскадах для обогащения регенерата урана позволяет рекомендовать область практической применимости подобных схем и их использовании в качестве способа получения обогащенного регенерированного урана.
  \item Разработаны рекомендации по использованию результатов работы для обогащения регенерированного урана в условиях однократного и многократного рецикла в различных видах топлива.
  \item  Представленные в работе результаты могут быть использованы в расчетных группах на предприятиях и организациях, связанных как с проектированием и построением разделительных каскадов, так и непосредственным производством изотопной продукции (АО «Уральский электрохимический комбинат», АО «Сибирский химический комбинат», АО «ТВЭЛ», АО «Восточно-Европейский головной научно-исследовательский и проектный институт энергетических технологий», АО «ПО «ЭХЗ» и др.).
  \item Разработан тренировочный программный комплекс для расчета каскада, нацеленного на возврат регенерированного урана. Код оформлен в виде лабораторной работы, которая затем внедрена в учебный процесс.
\end{enumerate}

{\methods} Для проведения расчетов использовались модельные каскады, а именно квазиидеальный каскад и его разновидность R-каскад. Рассматривался противоточный симметричный каскад ($\alpha=\beta=\sqrt{q}$). Применялись современные программные средства языка программирования python и подключаемые библиотеки, такие как SciPy, предназначенная для проведения инженерных расчетов, Matplotlib для визуализации результатов, и др..

{\defpositions}
\begin{enumerate}
  \item Результаты анализа физических закономерностей массопереноса компонентов смеси регенерированного урана в ординарном каскаде, позволяющие однозначно определить условия при которых возможно/невозможно обогащение регенерированного урана различного исходного состава в одиночном каскаде.
  \item Физико-математические модели, методики расчета и оптимизации модифицированных двойных и тройных каскадных схем для обогащения
  регенерата урана с одновременным выполнением условий на концентрации четных изотопов и максимальным использованием исходного материала.
  \item Методика выбора каскадной схемы обогащения регенерированного урана в условиях многократного рецикла, в зависимости от его исходного состава и принятых ограничений на концентрации четных изотопов.
\end{enumerate}
% В папке Documents можно ознакомиться в решением совета из Томского ГУ
% в~файле \verb+Def_positions.pdf+, где обоснованно даются рекомендации
% по~формулировкам защищаемых положений.

{\reliability} Надежность, достоверность и обоснованность научных положений и выводов, сделанных в диссертации, следует из корректности постановки задач, физической обоснованности применяемых приближений, использования в исследованиях методов, ранее примененных в аналогичных исследованиях, взаимной согласованности результатов исследования, а также из совпадения результатов численных экспериментов, полученных с помощью независимо разработанных методик как самим соискателем, так и другими исследователями. Корректность результатов вычислительных экспериментов гарантируется юнит-тестами, верифицирующими строгое выполнение заданных условий и соблюдение условий сходимости балансов (массовых и покомпонентных).

{\probation}
См. приложение А2.

{\contribution} Автор принимал активное участие в написании расчетных кодов, проведении вычислительных экспериментов, а также оформлении методики выбора каскадной схемы. Автором был разработан программный комплекс для сопровождения процесса принятия решений по выбору для заданной задачи каскада конфигурации, оптимальной по целевым критериям.

{\publications} См. приложение А1.

