{\actuality}
Построение ядерной энергетики нового типа, устойчивой к ресурсным ограничениям и предусматривающей решение проблемы обращения с радиоактивными отходами, связано с реакторами на быстрых нейтронах, нацеленными на воспроизводство делящегося материала -- энергетического  плутония. Однако, по оценкам \cite{andrianovaPerspektivnyeToplivnyeZagruzki2015}, в ближайшие десятилетия, по мере становления двухкомпонентной ядерно-энергетической системы, неизбежен переходный период, когда делящиеся материалы будут повторно использоваться в топливном цикле реакторов на тепловых нейтронах, так как они составляют основную часть парка энергоблоков.

% При этом, регенерированный уран в виде соединений $UO_3$ или $UNH$ имеет нулевую цену на выходе предприятия по переработку отработанного уранового топлива (Завода РТ) \cite{gresleyEnrichingRecyclingUranium1988}.

На сегодняшний день мире в состоянии эксплуатации и сооружения насчитывается порядка 500 ядерных энергоблоков, подавляющее большинство из которых представляют собой легководные реакторы \cite{PRISHome}. Такие реакторы потенциально могут использовать топливо, изготовленное на основе регенерированных материалов (урана и плутония). В связи с чем практический интерес представляет реализация замкнутого ядерного топливного цикла с многократным использованием накопленного в облученном/отработанном ядерном топливе (ОЯТ) урана и плутония.

Основным материалом отработавшего ядерного топлива является уран, составляющий $\approx$90-95\%, за вычетом конструкционных материалов. Так как регенерированный уран содержит $^{235}$U на уровне $\geq$0,85\%, то есть долю делящегося изотопа выше, чем в природном уране, дообогащать его на изотопно-разделительном производстве представляется экономически целесообразно \cite{NikipelovNikipelovSudby}.

Использование выделенного из ОЯТ регенерированного урана является наиболее достижимым на сегодняшний день направлением вовлечения регенерируемых материалов в топливный цикл энергетических реакторов. Выделенный из ОЯТ регенерированный уран может быть использован в составе топлива ВВЭР различными способами:
\begin{itemize}
  \item центрифужным дообогащением;
  \item включением (в некоторых случаях с предварительным дообогащением) в состав смешанного уран-плутониевого топлива типа MOX или REMIX.
\end{itemize}
 
Рецикл урана осложняет присутствие в изотопном составе регенерата ряда четных изотопов. В первую очередь, это неприродные $^{232}$U и $^{236}$U. Присутствие первого, являющегося опосредованным источником жесткого гамма-излучения, затрудняет обращение с регенератом, как на стадии обогащения, так и на стадии производства твэлов \cite{matveevUran232EgoVliyanie1985}. Влияние же второго сказывается в ухудшении размножающих свойств ядерного топлива, поскольку данный изотоп является паразитным поглотителем тепловых нейтронов \cite{dudnikovInfluence236UEfficacy2016}. Вдобавок, в регенерате, по сравнению с природным ураном, на порядок выше содержание $^{234}$U, что вносит в смесь регенерата дополнительную нежелательную радиоактивность. Ориентируясь на сегодняшние тенденции к увеличению длительности топливных циклов ВВЭР, которые связаны с повышением глубины выгорания топлива, следует принять во внимание вытекающий из этого рост содержания вредных четных изотопов в регенерате \cite{smirnovEvolutionIsotopicComposition2012}.

Так как регенерированный уран является многокомпонентной изотопной смесью, для которой необходимо учитывать ограничения на четные изотопы, решение проблемы рецикла урана упирается в необходимость корректировки его изотопного состава на стадии обогащения, чтобы производимое топливо отвечало стандартным требованиям. Поэтому, эффективное многократное вовлечение облученного урана в ядерный топливный цикл (ЯТЦ), тесно связано с поиском и дальнейшей разработкой каскадных схем, которые позволят производить из регенерата НОУ, удовлетворяющий стандартным спецификациям.
На сегодняшний день предложен ряд каскадных схем, которые могут быть применены для решения этой задачи, однако их возможности могут быть недостаточны при актуальных параметрах топливного цикла легководных реакторов. Поэтому возникает потребность поиска новых схем, которые могут быть применены для возврата урана в ядерный топливный цикл, с учетом новых условий. 

% Таким образом, учитывая принятое в ГК <<Росатом>> стратегическое решение о переходе к замкнутому ЯТЦ, решение перечисленных задач представляется актуальным для современной разделительной науки. 

{\aim} диссертационной работы является изучение физических закономерностей
молекулярно-селективного массопереноса в ординарных и многопоточных каскадах
для разделения многокомпонентных смесей с целью дальнейшего поиска
оптимальных условий обогащения регенерированного урана в подобных каскадах при
его многократном использовании в регенерированном ядерном топливе для реакторов на тепловых нейтронах. 

Для~достижения поставленной цели решены следующие {\tasks}:
\begin{enumerate}
  \item Анализ физических закономерностей массопереноса компонентов смеси
  регенерированного урана в ординарном каскаде.
  Выявление физических ограничений нахождения решения задачи обогащения регенерата произвольного изотопного
  состава в одиночном каскаде при одновременном выполнении условий на
  концентрации изотопов $^{232}$U, $^{234}$U и $^{236}$U в получаемом продукте – низкообогащенном уране, а также априорная оценка возможности решения этой задачи.
  \item Физическое обоснование принципов построения двойных каскадов,
  позволяющих корректировать изотопный состав регенерата по концентрациям
  изотопов $^{232}$U, $^{234}$U и $^{236}$U с одновременным расходованием максимального количества
  подлежащего обогащению регенерата при различных исходных концентрациях
  четных изотопов в нем.
  \item Обоснован способ эффективной «утилизации» загрязненной четными
  изотопами фракции, возникающей в двойных каскадах, с учетом полной или
  частичной подачи данной фракции: в третий каскад с предварительным
  перемешиванием ее с природным, обедненным и/или низкообогащенным ураном; в отдельный двойной каскад, осуществляющий наработку низкообогащенного урана для последующей топливной кампании реактора.
  \item Разработаны методы сравнения исследуемых каскадных схем, а также расчетные методики для оптимизации параметров каскадных схем.
  \item Изучены физические закономерности изменения изотопного состава регенерата, а также
  интегральных характеристик модифицированных двойных каскадов и тройных
  каскадов при обогащении регенерированного урана с различным исходным
  содержанием четных изотопов в питающей смеси.
  % \item Обобщение и систематизация подходов к выбору каскадной схемы, позволяющих
  % эффективное обогащение регенерированного урана в условиях однократного и
  % многократного рецикла.
  % \item Определение физических закономерностей изменения изотопного состава
  % регенерированного урана и параметров модифицированного двойного каскада для
  % его дообогащения при многократном рецикле урана (отдельно и совместно с
  % плутонием) в топливе реакторов типа ВВЭР.
\end{enumerate}


{\novelty}
\begin{enumerate}
  \item Впервые предложены модификации двойных каскадов, позволяющих корректировать
  изотопный состав регенерата по концентрациям изотопов $^{232}$U, $^{234}$U и $^{236}$U с одновременным расходованием полного количества подлежащего обогащению регенерата при различных исходных концентрациях четных изотопов в нем и других внешних условиях.
  \item Обоснованы физические принципы построения тройных каскадных схем для максимального вовлечения исходного регенерированного урана для воспроизводства топлива реакторов на тепловых нейтронах.
  \item Выполнены оригинальные исследования по изучению физических закономерностей изменения изотопного состава регенерата и интегральных характеристик модифицированных двойных и тройных каскадах при обогащении регенерированного урана с различным исходным содержанием четных изотопов.
  \item Разработаны методы расчетов каскадных схем, позволяющих решить задачу возврата регенерированного урана в топливный цикл в условиях многократного рециклирования.
  \item Разработан обобщенный подход к выбору каскадной схемы для эффективного обогащения регенерированного урана в условиях однократного и многократного рецикла.
  \item Разработка методик оптимизации систем каскадов (двойного и тройного каскадов) для обогащения регенерата урана по различным критериям эффективности, таким как:
  % \begin{enumerate}
  %   \item расход природного урана в цикле;
  %   \item затраты работы разделения в цикле;
  %   \item доля потерь $^{235}$U в каскадной схеме;
  %   \item доля потерь $^{235}$U из исходного регенерата;
  %   \item доля газовых центрифуг в схеме, в которых превышена предельно допустимая концентрация по $^{232}$U.
  % \end{enumerate}
  \item Разработка подхода к утилизации высокоактивного «нештатного» отхода, образующегося в процессе обогащения регенерированного урана в двойном каскаде.
  \item Определение физических закономерностей изменения изотопного состава регенерированного урана и параметров каскадных схем (в модифицированном двойном и тройном каскаде) для его дообогащения при многократном рецикле урана (отдельно и совместно с плутонием) в топливе реакторов типа ВВЭР.
\end{enumerate}

{\influence} 
\begin{enumerate}
  \item Проведенный анализ физических закономерностей массопереноса компонентов смеси регенерированного урана в ординарном каскаде позволяет однозначно определить условия при которых возможно/невозможно получение необходимого количества конечного продукта на основе регенерированного урана различного исходного состава путем обогащения в одиночном каскаде.
  \item Разработанные модификации двойных и тройных каскадов позволяют эффективно решать задачу обогащения регенерированного урана с одновременным выполнением ограничений на концентрации четных изотопов и максимальным вовлечением исходного регенерата.
  % \item Проведенный анализ результатов расчетного моделирования молекулярно-селективного массопереноса в модифицированных двойных и тройных каскадах для обогащения регенерата урана выявляет область практической применимости подобных схем для получения НОУ-продукта на основе регенерированного урана.
  \item Предложенные способы оптимизации построения каскадных схем двойного и тройного каскадов позволяют находить наиболее эффективные с точки зрения таких критериев, как расход работы разделения, расход природного урана, степень извлечения $^{235}$U, конфигурации каскадов для возврата регенерированного урана в цикл.
  \item Разработаны рекомендации по использованию результатов работы для обогащения регенерированного урана в условиях однократного и многократного рецикла в различных видах топлива. Представленные в работе результаты могут быть использованы в расчетных группах на предприятиях и организациях, связанных как с проектированием и построением разделительных каскадов, так и непосредственным производством изотопной продукции (АО «Уральский электрохимический комбинат», АО «Сибирский химический комбинат», АО «ТВЭЛ», АО «Восточно-Европейский головной научно-исследовательский и проектный институт энергетических технологий», АО «ПО «ЭХЗ» и др.). Предложенные методики расчета могут лечь в основу технико-экономического анализа использования восстановленного урана для получения низкообогащенного урана, отвечающего требуемым качествам.  
  % \item Разработан тренировочный программный комплекс для расчета каскада, нацеленного на возврат регенерированного урана. Код оформлен в виде лабораторной работы, которая внедрена в учебный процесс.
\end{enumerate}


{\methods}.
Исследование проводит систематизацию научно-технической литературы, посвященной заявленной теме.
Применены подходы, известные в современной теоретической физике, и в частности, в теории разделения изотопов в каскадах.
В ходе работы обоснованы принципы построения анализируемых каскадов, и проведено математическое моделирование каскадных схем.
Для проведения расчетов использованы схемы модельных каскадов (квазиидеальный каскад и его разновидность R-каскад, для которого выполняется условие несмешивания относительных концентраций пары выбранных компонентов). Моделирование процессов разделения смесей изотопов урана проводили с использованием разработанных в ходе выполнения работы специализированных компьютерных программ. Применены современные программные средства языков программирования Julia и Python и подключаемых библиотек, таких как NLopt, Optim, ScyPy, предназначенных для решения систем нелинейных уравнений и оптимизационных процедур, Plots.jl для визуализации результатов.

{\defpositions}
\begin{enumerate}
  \item Результаты анализа физических закономерностей массопереноса компонентов смеси регенерированного урана в ординарном каскаде, позволяющие однозначно определить условия при которых возможно/невозможно получение необходимого количества конечного продукта на основе регенерированного урана различного исходного состава путем обогащения в одиночном каскаде.
  \item Физико-математические модели, методики расчета и оптимизации модифицированных двойных и тройных каскадных схем для обогащения регенерата урана с одновременным выполнением условий на концентрации четных изотопов и максимальным вовлечением исходного материала.
  \item Методика выбора каскадной схемы обогащения регенерированного урана в условиях многократного рецикла, в зависимости от его исходного состава и принятых ограничений на концентрации четных изотопов.
\end{enumerate}

{\reliability}.
Надежность, достоверность и обоснованность научных положений и выводов, сделанных в диссертации, следует из корректности постановки задач, физической обоснованности применяемых приближений, использования методов, ранее примененных в аналогичных исследованиях, взаимной согласованности результатов, а также из совпадения результатов численных экспериментов. Корректность результатов вычислительных экспериментов гарантируется тестами и операторами проверки соответствия ограничениям, верифицирующими строгое выполнение заданных условий и соблюдение условий сходимости балансов (массовых и покомпонентных).

% {\probation}
% См. приложение А2.

{\contribution} Автор принимал активное участие разработке каскадных схем, написании расчетных кодов, проведении вычислительных экспериментов, а также в обработке и в анализе результатов численных экспериментов. Автор разработал расчетные коды, реализующие новые подходы к оптимизации рассматриваемых схем.

% {\publications} 
% См. приложение А1.

