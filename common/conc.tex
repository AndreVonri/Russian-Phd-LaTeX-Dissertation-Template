По итогам выполнения диссертационной работы сделаны следующие научные и практические выводы.
\begin{enumerate}[label=\Roman*.]
\item Предложен модифицированный двойной каскад с НОУ-разбавителем из природного урана, применимый  для обогащения регенерированного урана в условиях многократного рецикла урана в топливе легководных реакторов и позволяющий получить продукт, отвечающий всем требованиям на концентрации четных изотопов. 

\begin{enumerate}
    \item На основе теории квазиидеального каскада разработаны методики расчета и оптимизации предложенной каскадной схемы по различным критериям эффективности (затраты работы разделения, расход природного урана, степень извлечения $^{235}$U из регенерата, степень извлечения $^{235}$U из всех питающих потоков схемы). Показано, что эффективность предложенной каскадной схемы по тому или иному критерию зависит от выбранного диапазона изменения концентрации $^{235}$U в потоке отбора каскада II --- $P_2$. Наиболее выгодные с точки зрения выбранных критериев эффективности наборы параметров каскадной схемы лежат в области, где $C_{235,{P_2}} > 20\%$. Это означает, что при практической реализации модифицированного двойного каскада целесообразно рассматривать возможность получения в отдельных потоках такой схемы концентраций $^{235}$U, превышающих 20\%, и, в первую очередь, в потоке $P_2$. 
    \item Анализ эффективности предложенной каскадной схемы с точки зрения потерь $^{235}$U показал, что схема обеспечивает экономию природного урана по сравнению с открытым топливным циклом на уровне 15-20\% в зависимости от исходного изотопного состава регенерата. Это превышает аналогичные показатели для простейших разбавляющих схем практически вдвое.
    \item Предложенная схема позволяет полностью решить задачу обогащения регенерата в широком диапазоне внешних условий и ограничений, что создает базис для ее практической реализации и поиска наиболее эффективных режимов ее работы.
\end{enumerate}

\item Показано, что модификации ординарного каскада для обогащения и разбавления регенерированного урана принципиально не решают задачу обогащения регенерированного урана при одновременном выполнении условий на концентрации четных изотопов в товарном НОУ и обеспечения расходования заданной массы регенерата на получение этого НОУ для составов регенерата с исходным содержанием четных изотопов, превышающим предельные значения для товарного НОУ. 

Основная причина невозможности решения задачи состоит в том, что в рассматриваемых схемах число свободных параметров оказывается меньшим, чем число условий, которые необходимо одновременно удовлетворить. В результате такие схемы могут обеспечить решение задачи только в частных случаях, когда в обогащение поступает регенерированный уран с исходными концентрациями четных изотопов ниже предельных значений для товарного НОУ.

\item Обоснованы способы вовлечения загрязненной четными изотопами фракции, возникающей в двойных каскадах при очистке от $^{232}$U, с учетом полной или частичной подачи данной фракции: а) в отдельный двойной каскад, осуществляющий наработку низкообогащенного урана для последующей топливной кампании реактора; б) перемешивании этой фракции с потоками обедненного урана и низкообогащенного урана для получения дополнительной массы товарного НОУ; в) в третий каскад с предварительным перемешиванием ее с природным, обедненным и/или низкообогащенным ураном. Для каждого из способов проанализированы их достоинства и недостатки, и вытекающие из них области применения, а также рассчитаны получаемые преимущества относительно открытого ЯТЦ.
 
\item Результаты работы создают основу для проведения дальнейшего технико-экономического анализа каждой из схем на основе их интегральных показателей, таких как расход природного урана, затраты работы разделения, потери $^{235}$U в цикле в контексте всей цепочки ядерного топливного цикла, а также с учетом возникающих в этой цепочке изменений при использовании регенерата урана по отношению к открытому топливному циклу. Полученные в диссертации результаты дополняют теорию каскадов для разделения изотопов. В частности, предложенные в работе методики оптимизации двойных и тройных каскадов могут быть адаптированы к случаю разделения многокомпонентных смесей неурановых изотопов в каскадах центрифуг.

\end{enumerate}
