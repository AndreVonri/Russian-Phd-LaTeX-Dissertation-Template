\begin{enumerate}[label=\Roman*.]
\item Результаты вычислительных экспериментов, проведенных для различных модификаций ординарного каскада для обогащения и разбавления регенерированного урана, показали:
\begin{enumerate}
  \item Основанные на ординарном каскаде схемы принципиально не решают задачу обогащения регенерированного урана при одновременном выполнении условий на концентрации четных изотопов в товарном НОУ и обеспечения расходования заданной массы регенерата на получение этого НОУ для составов регенерата с исходным содержанием чётных изотопов, превышающим предельные значения для товарного НОУ. 
  \item Основная причина невозможности решения задачи состоит в том, что в рассматриваемых схемах число свободных параметров оказывается меньшим, чем число условий, которые необходимо одновременно удовлетворить. В результате такие схемы могут обеспечить решение задачи только в некоторых частных случаях, например, когда в обогащение поступает регенерированный уран с относительно низкими исходными концентрациями четных изотопов, то есть выделенный из однократно облученного уранового топлива -- так называемый регенерат первого рецикла.
\end{enumerate}

\item Предложена модификация двойного каскада с НОУ-разбавителем из природного урана, применимая для обогащения регенерированного урана в условиях многократного рецикла урана в топливе легководных реакторов и позволяющая получить продукт, отвечающий всем требованиям на концентрации чётных изотопов. Достоинствами схемы является возможность частичного отделения легких изотопов $^{232}$U, $^{234}$U от $^{235}$U, а также обособленность участков обогащения регенерированного урана и природного урана. Последнее обеспечивает большую вариативность в возможностях практической реализации подобной схемы, а также позволяет избежать загрязнения значительной части разделительного оборудования изотопами $^{232,234}$U.

Процесс обогащения регенерата урана различного исходного состава в предложенной схеме смоделирован с использованием теории квазиидеального каскада. Разработаны методики расчета и оптимизации предложенной каскадной схемы по различным критериям эффективности (затраты работы разделения, расход природного урана, степень извлечения $^{235}$U из регенерата, степень извлечения $^{235}$U из всех питающих потоков схемы). Показано, что схема позволяет полностью решить задачу обогащения регенерата в широком диапазоне внешних условий, что создает базис для практической её реализации и поиска наиболее эффективных режимов ее работы.

Анализ эффективности предложенной каскадной схемы с точки зрения потерь $^{235}$U показал, что схема позволяет извлечь более 85\% от массы $^{235}$U из исходного регенерированного урана, поступившего на обогащение. Это обеспечивает экономию природного урана по сравнению с открытым топливным циклом на уровне 15-20\% в зависимости от исходного изотопного состава регенерата. Таким образом, эта схема превышает аналогичные показатели для простейших разбавляющих схем практически вдвое.

\item Обоснованы способы эффективной «утилизации» загрязненной четными изотопами фракции, возникающей в двойных каскадах при очистке от $^{232}$U, с учетом полной или частичной подачи данной фракции: а) в третий каскад с предварительным перемешиванием ее с природным, обедненным и/или низкообогащенным ураном; б) в отдельный двойной каскад, осуществляющий наработку низкообогащенного урана для последующей топливной кампании реактора; в) перемешивании этой фракции с потоками обедненного урана и низкообогащенного урана для получения дополнительной массы товарного НОУ. Для каждого из предложенных способов проведены вычислительные эксперименты, анализ результатов которых позволил сформулировать достоинства и недостатки каждого из способов и очертить возможные области их применения.
\begin{enumerate}
    \item Предложенная модификация тройного каскада позволяет утилизировать фракцию отхода каскада II -- поток $P_2$, обеспечивая при этом отсутствии нештатных отходов, кроме потоков обедненного урана с невысоким содержанием чётных изотопов (в несколько раз ниже допустимых в НОУ-продукте значений). Для такой каскадной схемы разработана методика расчёта и оптимизации её параметров по различным критериям эффективности, что позволило показать, что схема может одновременно обеспечить экономию как природного урана, так и затрат работы разделения по отношению к открытому ЯТЦ даже в случае обогащения регенерата с высоким содержанием чётных изотопов, свойственному изотопным составам при многократном рециклировании.
    \item Характерными недостатками схемы двойного каскада с НОУ-разбавителем с возвратом потока загрязненной $^{232}$U фракции в цикл является возврат значительной части четных изотопов на вход каскадной схемы. Это приводит к тому, что при повторении такого процесса концентрации четных изотопов в исходном регенерате могут существенно возрастать (до нескольких раз), тем самым снижая эффективность обогащения регенерата в целом. Следовательно, такой подход применим для 1-3 таких возвратов.
    \item Схема утилизации загрязненной $^{232}$U фракции через ее разбавление обедненным ураном и НОУ из природного урана обеспечивает возможность наработки дополнительной массы отвечающего всем требованиям товарного продукта. При этом в схеме отсутствуют дополнительные каскады, что исключает дополнительные затраты работы разделения. Данный способ утилизации пригоден при обогащении различных вариантов составов исходного регенерата, что делает его перспективным в условиях многократного рецикла.
\end{enumerate}

\item Результаты работы дают основу для проведения дальнейшего технико-экономического анализа каждой из схем на основе их интегральных показателей, таких как расход природного урана, затраты работы разделения, потери $^{235}$U в цикле в контексте всей цепочки ядерного топливного цикла, а также с учетом возникающих в этой цепочке изменений при использовании регенерата урана по отношению к открытому топливному циклу. Помимо этого, необходима проработка технологических проблем каждой из схем, в частности, с точки зрения возможности эксплуатации и обслуживания оборудования в условиях работы с материалами, имеющими более высокую, чем природный уран удельную активность. 

\end{enumerate}
