Проанализирована проблема и известные способы обогащения регенерированного урана в каскадах центрифуг в условиях многократного рецикла урана в топливе легководных реакторов на тепловых нейтронах. По итогам выполнения диссертационной работы сделаны следующие научные и практические выводы.
\begin{enumerate}[label=\Roman*.]
\item Показано, что модификации ординарного каскада для обогащения и разбавления регенерированного урана принципиально не решают задачу обогащения регенерированного урана при одновременном выполнении условий на концентрации четных изотопов в товарном НОУ и обеспечения расходования заданной массы регенерата на получение этого НОУ для составов регенерата с исходным содержанием чётных изотопов, превышающим предельные значения для товарного НОУ. 

Основная причина невозможности решения задачи состоит в том, что в рассматриваемых схемах число свободных параметров оказывается меньшим, чем число условий, которые необходимо одновременно удовлетворить. В результате такие схемы могут обеспечить решение задачи только в частных случаях, когда в обогащение поступает регенерированный уран с исходными концентрациями четных изотопов ниже предельных значений для товарного НОУ.

\item Предложена модификация двойного каскада с НОУ-разбавителем из природного урана, применимая для обогащения регенерированного урана в условиях многократного рецикла урана в топливе легководных реакторов и позволяющая получить продукт, отвечающий всем требованиям на концентрации чётных изотопов. Достоинствами схемы является возможность частичного отделения легких изотопов $^{232}$U, $^{234}$U от $^{235}$U, а также обособленность участков обогащения регенерированного урана и природного урана. Последнее обеспечивает большую вариативность в возможностях практической реализации подобной схемы, а также позволяет избежать загрязнения значительной части разделительного оборудования изотопами $^{232,234}$U. 

На основе теории квазиидеального каскада разработаны методики расчета и оптимизации предложенной каскадной схемы по различным критериям эффективности (затраты работы разделения, расход природного урана, степень извлечения $^{235}$U из регенерата, степень извлечения $^{235}$U из всех питающих потоков схемы). Показано, что эффективность предложенной каскадной схемы по тому или иному критерию зависит от выбранного диапазона изменения концентрации $^{235}$U в потоке отбора каскада II -- $P_2$. Наиболее выгодные с точки зрения зрения основных критериев эффективности наборы параметров каскадной схемы лежат в области, где $C_{235,{P_2}} > 20\%$. Это означает, что при практической реализации модифицированного двойного каскада целесообразно рассматривать возможность получения в отдельных потоках такой схемы концентраций $^{235}$U, превышающих 20\%, и, в первую очередь, в потоке $P_2$. 

Анализ эффективности предложенной каскадной схемы с точки зрения потерь $^{235}$U показал, что схема позволяет извлечь более 85\% от массы $^{235}$U из исходного регенерированного урана, поступившего на обогащение. Это обеспечивает экономию природного урана по сравнению с открытым топливным циклом на уровне 15-20\% в зависимости от исходного изотопного состава регенерата. Таким образом, эта схема превышает аналогичные показатели для простейших разбавляющих схем практически вдвое.

По резульататам серии вычислительных экспериментов показано, что предложенная схема позволяет полностью решить задачу обогащения регенерата в широком диапазоне внешних условий, что создает базис для практической её реализации и поиска наиболее эффективных режимов ее работы.

\item Обоснованы способы «утилизации» загрязненной четными изотопами фракции, возникающей в двойных каскадах при очистке от $^{232}$U, с учетом полной или частичной подачи данной фракции: а) в третий каскад с предварительным перемешиванием ее с природным, обедненным и/или низкообогащенным ураном; б) в отдельный двойной каскад, осуществляющий наработку низкообогащенного урана для последующей топливной кампании реактора; в) перемешивании этой фракции с потоками обедненного урана и низкообогащенного урана для получения дополнительной массы товарного НОУ. Выявлены достоинства и недостатки каждого из способов, что позволяет обозначить возможные области их применения.
\begin{enumerate}
    \item Предложенная модификация тройного каскада позволяет утилизировать фракцию отхода каскада II -- поток $P_2$, обеспечивая при этом отсутствии нештатных отходов, кроме потоков обедненного урана с невысоким содержанием чётных изотопов (в несколько раз ниже допустимых в НОУ-продукте значений). Для такой каскадной схемы разработана методика расчёта и оптимизации её параметров по различным критериям эффективности, что позволило показать, что схема может одновременно обеспечить экономию как природного урана, так и затрат работы разделения по отношению к открытому ЯТЦ даже в случае обогащения регенерата с концентрацией $^{232}$U выше предельных значений, что характерно для регенерата в условиях его многократного рециклирования.
    \item Характерными недостатками схемы двойного каскада с НОУ-разбавителем с возвратом потока загрязненной $^{232}$U фракции в цикл является возврат значительной части четных изотопов на вход каскадной схемы. Это приводит к тому, что при повторении такого процесса концентрации четных изотопов в исходном регенерате могут существенно возрастать (до нескольких раз), тем самым снижая эффективность обогащения регенерата в целом. Такой подход целесообразно применять для 1-3 таких возвратов.
    \item Схема утилизации загрязненной $^{232}$U фракции через ее разбавление обедненным ураном и НОУ из природного урана обеспечивает возможность наработки дополнительной массы отвечающего всем требованиям товарного продукта. При этом в схеме отсутствуют дополнительные затраты работы разделения. Данный способ утилизации пригоден при обогащении различных вариантов составов исходного регенерата, что делает его перспективным в условиях многократного рецикла.
\end{enumerate}

\item Результаты работы создают основу для проведения дальнейшего технико-экономического анализа каждой из схем на основе их интегральных показателей, таких как расход природного урана, затраты работы разделения, потери $^{235}$U в цикле в контексте всей цепочки ядерного топливного цикла, а также с учетом возникающих в этой цепочке изменений при использовании регенерата урана по отношению к открытому топливному циклу.  

\item Полученные в диссертации результаты дополняют теорию каскадов для разделения изотопов и могут быть использованы при подготовке студентов по профильным специальностям, а также для повышения квалификации персонала разделительных производств. Предложенные в работе методики оптимизации двойных и тройных каскадов могут быть адаптированы к случаю разделения многокомпонентных смесей неурановых изотопов в каскадах центрифуг.

\end{enumerate}
