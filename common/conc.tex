В работе теоретически обоснованы способы обогащения регенерированного урана с одновременной коррекцией его изотопного состава по содержанию четных изотопов, основанных на модификациях двойных каскадов. Показана применимость предложенных модификаций двойного каскада в условиях обогащения регенерированного урана с исходным содержанием четных изотопов выше допустимых пределов, что в несколько раз превышает содержание указанных изотопов в составах регенерата ранее рассмотренных в теоретических исследованиях. Это означает возможность успешного использования предложенных подходов в условиях многократного рецикла, когда концентрации четных изотопов возрастают от рецикла к рециклу.

По результатам проведенного исследования можно сформулировать следующие конкретные выводы.

\begin{enumerate}
\item В работе предложена модификация двойного каскада с НОУ-разбавителем из природного урана, применимая для обогащения регенерированного урана в условиях многократного рецикла урана в топливе легководных реакторов и позволяющая получить продукт, отвечающий всем требованиям на концентрации четных изотопов для регенерата различного исходного состава. Достоинствами схемы является возможность частичного отделения легких изотопов $^{232}$U, $^{234}$U от $^{235}$U, а также обособленность участков обогащения регенерированного урана и природного урана. Последнее обеспечивает большую вариативность в возможностях практической реализации подобной схемы, а также позволяет избежать загрязнения значительной части разделительного оборудования четными изотопами урана.


Процесс обогащения регенерата урана различного исходного состава в предложенной схеме смоделирован с использованием теории квазиидеального каскада. Разработаны оригинальные методики расчета и оптимизации предложенной каскадной схемы по различным критериям эффективности, таким как минимум расхода природного урана, минимум затрат работы разделения на получение конечного продукта, максимум степени извлечения целевого изотопа $^{235}$U из исходного регенерата и др. Проведена серия вычислительных экспериментов, позволившая оценить ключевые интегральные характеристики предложенной модификации двойного каскада (удельный расход природного урана, затраты работы разделения) в широком диапазоне изменение как ее параметров, так и внешних условий. Анализ результатов проведенных вычислительных экспериментов показал, что схема оказывается устойчивой и позволяет полностью решить задачу в широком диапазоне внешних условий. Например, в случае уменьшения предельно допустимой концентрации $^{232}$U в товарном НОУ или кратного (до трех раз) возрастания массы исходного регенерированного урана, которую нужно израсходовать для получения заданной массы товарного продукта. Анализ полученных результатов создает базис для дальнейшей практической реализации подобной схемы и поиска наиболее эффективных режимов ее работы.

Анализ эффективности предложенной каскадной схемы с точки зрения потерь $^{235}$U показал, что схема позволяет извлечь более 85\% от массы $^{235}$U из исходного регенерированного урана, поступившего на обогащение. Это обеспечивает экономию природного урана по сравнению с открытым топливным циклом на уровне 15-20\% в зависимости от исходного изотопного состава регенерата. Таким образом, эта схема превышает аналогичные показатели для простейших разбавляющих схем практически вдвое.
\item Обоснованы способы эффективной «утилизации» загрязненной четными изотопами фракции, возникающей в двойных каскадах при очистке от $^{232}$U, с учетом полной или частичной подачи данной фракции: а) в третий каскад с предварительным перемешиванием ее с природным, обедненным и/или низкообогащенным ураном; б) в отдельный двойной каскад, осуществляющий наработку низкообогащенного урана для последующей топливной кампании реактора; в) перемешивании этой фракции с потоками обедненного урана и низкообогащенного урана для получения дополнительной массы товарного НОУ. Для каждого из предложенных способов проведены вычислительные эксперименты, анализ результатов которых позволил сформулировать достоинства и недостатки каждого из способов и очертить возможные области их применения.
В качестве основных выводов по этой части приведем следующие:
\begin{enumerate}
\item предложенная модификация тройного каскада позволяет утилизировать фракцию отхода каскада II -- поток $P_2$, обеспечивая при этом отсутствии нештатных отходов, кроме потоков обедненного урана с относительно невысоким содержанием чётных изотопов, что в несколько раз ниже предельных значений. Для такой каскадной схемы разработана оригинальная методика расчёта и оптимизации её параметров по различным критериям эффективности, что позволило показать, что схема может одновременно обеспечить экономию как природного урана, так и затрат работы разделения по отношению к открытому ЯТЦ даже в случае обогащения регенерата с относительно высоким содержанием чётных изотопов.
\item Характерными недостатками схемы двойного каскада с НОУ-разбавителем с возвратом потока загрязненной $^{232}$U фракции в цикл является возврат значительной части четных изотопов на вход каскадной схемы. Это приводит к тому, что при повторении такого процесса несколько раз концентрации четных изотопов в исходном регенерате могут существенно возрастать (до нескольких раз), тем самым снижая эффективность обогащения регенерата в целом. Следовательно, такой подход применим для 1-3 таких возвратов, далее целесообразно рассмотреть возможность реализации одного из других предложенных способов утилизации отхода.
\item Схема утилизации загрязненной $^{232}$U фракции через ее разбавление обедненным ураном и НОУ из природного урана обеспечивает возможность наработки дополнительной массы отвечающего всем требованиям товарного продукта. При этом в схеме отсутствуют дополнительные каскады, что исключает дополнительные затраты работы разделения. Показано, что данный способ утилизации пригоден при обогащении различных вариантов составов исходного регенерата, что делает его перспективным в условиях многократного рецикла.
\end{enumerate}

\item Результаты работы дают основу для проведения дальнейшего технико-экономического анализа каждой из схем на основе их интегральных показателей, таких как расход природного урана, затраты работы разделения, потери $^{235}$U в цикле в контексте всей цепочки ядерного топливного цикла, а также с учетом возникающих в этой цепочке изменений при использовании регенерата урана по отношению к открытому топливному циклу. Помимо этого, необходима проработка технологических проблем каждой из схем, в частности, с точки зрения возможности эксплуатации и обслуживания оборудования в условиях работы с материалами, имеющими более высокую, чем природный уран удельную активность. 
\end{enumerate}
