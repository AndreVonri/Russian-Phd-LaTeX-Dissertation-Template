Теоретически обоснованы способы обогащения регенерированного урана с одновременной коррекцией его изотопного состава по содержанию чётных изотопов, основанных на модификациях двойных каскадов. Показана применимость предложенных модификаций двойного каскада в условиях обогащения регенерированного урана с исходным содержанием чётных изотопов выше допустимых пределов, что в несколько раз превышает содержание указанных изотопов в составах регенерата ранее рассмотренных в теоретических исследованиях. Это означает возможность успешного использования предложенных подходов в условиях многократного рецикла, когда концентрации чётных изотопов возрастают от рецикла к рециклу.

По результатам проведенного исследования можно сформулировать следующие конкретные выводы.

\begin{enumerate}
\item В работе предложена модификация двойного каскада с НОУ-разбавителем из природного урана, применимая для обогащения регенерированного урана в условиях многократного рецикла урана в топливе легководных реакторов и позволяющая получить продукт, отвечающий всем требованиям на концентрации четных изотопов для регенерата различного исходного состава. Достоинствами схемы является возможность частичного отделения лёгких изотопов $^{232}$U, $^{234}$U от $^{235}$U, а также обособленность участков обогащения регенерированного урана и природного урана. Последнее обеспечивает большую вариативность в возможностях практической реализации подобной схемы, а также позволяет избежать загрязнения значительной части разделительного оборудования чётными изотопами урана.


Процесс обогащения регенерата урана различного исходного состава в предложенной схеме смоделирован с использованием теории квазиидеального каскада. Разработаны оригинальные методики расчёта и оптимизации предложенной каскадной схемы по различным критериям эффективности, таким как минимум расхода природного урана, минимум затрат работы разделения на получение конечного продукта, максимум степени извлечения целевого изотопа $^{235}$U из исходного регенерата и др. Проведена серия вычислительных экспериментов, позволившая оценить ключевые интегральные характеристики предложенной модификации двойного каскада (удельный расход природного урана, затраты работы разделения) в широком диапазоне изменение как её параметров, так и внешних условий. Анализ результатов проведенных вычислительных экспериментов показал, что схема оказывается устойчивой в случаях, когда внешние ограничения <<ужесточаются>>. Например, уменьшается предельно допустимая концентрация $^{232}$U в товарном НОУ или кратно (до трех раз) возрастает масса исходного регенерированного урана, которую нужно израсходовать для получения заданной массы товарного продукта. Анализ полученных результатов создаёт базис для дальнейшей практической реализации подобной схемы и поиска наиболее эффективных режимов её работы.

Анализ эффективности предложенной каскадной схемы с точки зрения потерь $^{235}$U показал, что схема позволяет извлечь более 85\% от массы $^{235}$U из исходного регенерированного урана, поступившего на обогащение. Это обеспечивает экономию природного урана по сравнению с открытым топливным циклом на уровне 15-20\% в зависимости от исходного изотопного состава регенерата. Таким образом, эта схема превышает аналогичные показатели для простейших разбавляющих схем практически вдвое.
\item Обоснован способ эффективной «утилизации» загрязненной четными изотопами фракции, возникающей в двойных каскадах при очистке от $^{232}$U, с учетом полной или частичной подачи данной фракции: а) в третий каскад с предварительным перемешиванием ее с природным, обедненным и/или низкообогащенным ураном; б) в отдельный двойной каскад, осуществляющий наработку низкообогащенного урана для последующей топливной кампании реактора. Для каждого из предложенных способов проведены вычислительные эксперименты, анализ результатов которых позволил сформулировать достоинства и недостатки каждого из способов и очертить возможные области их применения.
В качестве основных выводов по этой части приведем следующие:
\begin{enumerate}
\item Характерными недостатками схемы двойного каскада с НОУ-разбавителем с возвратом потока загрязнённой $^{232}$U фракции в цикл является возврат значительной части четных изотопов на вход каскадной схемы. Это приводит к тому, что при повторении такого процесса несколько раз концентрации чётных изотопов в исходном регенерате могут существенно возрастать (до нескольких раз), тем самым снижая эффективность обогащения регенерата в целом. Следовательно, такой подход применим для 1-3 таких возвратов, далее целесообразно рассмотреть возможность реализации одного из других предложенных способов утилизации отхода.
\item Характерным недостатком схемы тройного каскада для утилизации загрязнённой $^{232}$U фракции является увеличение затрат работы разделения по отношению к другим рассмотренным модификациям, возникающее при обогащении разбавленного обедненным ураном загрязненного четными изотопами отхода. Анализ результатов серии вычислительных экспериментов, проведенных для данной схемы позволяет говорить, что она хорошо применима для составов регенерированного урана, когда исходное содержание $^{232}$U ещё не превысило предельно допустимых значений для продукта. Иными словами, такой подход может подойти для обогащения регенерата 1-го и 2-го рецикла, позволяя вернуть в цикл более 90\% массы $^{235}$U из регенерированного урана. 
\item Схема утилизации загрязнённой $^{232}$U фракции через её разбавление обедненным ураном и НОУ из природного урана обеспечивает возможность наработки дополнительной массы отвечающего всем требованиям товарного продукта. При этом в схеме отсутствуют дополнительные каскады, что исключает дополнительные затраты работы разделения. Показано, что данный способ утилизации пригоден при обогащении различных вариантов составов исходного регенерата, что делает его перспективным в условиях многократного рецикла.
\end{enumerate}

\item Результаты работы дают основу для проведения дальнейшего технико-экономического анализа каждой из схем на основе их интегральных показателей, таких как расход природного урана, затраты работы разделения, потери $^{235}$U в цикле в контексте всей цепочки ядерного топливного цикла, а также с учетом возникающих в этой цепочке изменений при использовании регенерата урана по отношению к открытому топливному циклу. Помимо этого, необходима проработка технологических проблем каждой из схем, в частности, с точки зрения возможности эксплуатации и обслуживания оборудования в условиях работы с материалами, имеющими более высокую, чем природный уран удельную активность. 
\end{enumerate}
