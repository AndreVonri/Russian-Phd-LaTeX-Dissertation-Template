%% Согласно ГОСТ Р 7.0.11-2011:
%% 5.3.3 В заключении диссертации излагают итоги выполненного исследования, рекомендации, перспективы дальнейшей разработки темы.
%% 9.2.3 В заключении автореферата диссертации излагают итоги данного исследования, рекомендации и перспективы дальнейшей разработки темы.

В результате проведения диссертационной работы разработаны варианты каскадных схем, позволяющие решить задачу обогащения регенерированного урана в условиях его многократного рецикла при условиях, характерных топливному циклу современных легководных реакторов российского дизайна и международным спецификациям:

\begin{enumerate}
  \item Схема двойного каскада с НОУ-разбавителем;
  \item Схема двойного каскада с НОУ-разбавителем с возвратом потока $P_2$ в цикл;
  \item Схема тройного каскада с НОУ-разбавителем и дополнительным разбавителем потока $P_2$, возвращаемого в цикл
  \item Схема независимой утилизации побочного продукта легкой фракции второго каскада схемы двойного каскада с НОУ-разбавителем.
\end{enumerate}

Для каждой из предложенных схем разработаны оригинальные методики расчета и оптимизации ее параметров по критерию минимума суммарного потока каскадной схемы, основанная на использовании современных методов условной оптимизации функций многих переменных. С использованием разработанных методик расчета и оптимизации предложенных каскадных схем продемонстрирована возможность их использования для обогащения регенерированного урана в условиях многократного рецикла на примере взятого из литературы изотопного состава регенерата урана с повышенным содержанием четных изотопов и отвечающего пятому рециклу в топливе ВВЭР.

-	характерным недостатком схемы 1 является наличие отхода с высоким содержанием четных изотопов (на 1-2 порядка выше, чем пределы для товарного НОУ) и 235U (до 20% или до 90%, в зависимости от выбранного режима работы каскадной схемы). Одним из вариантом обращения с подобным отходом может стать его перемешивание с отвалом первого каскада при обогащении регенерата. Оценки показали, что в этом случае возможно получить обедненный уран с приемлемым содержанием 232U (не выше 5·10-7%).
-	характерными недостатками схемы 2 являются: 1) монотонный рост концентраций четных изотопов от рецикла к рециклу как в исходном регенерате, так и в получаемом товарном НОУ, вследствие возврата значительной части четных изотопов на вход каскадной схемы; 2) рост потерь 235U от первоначальной массы, поступившей на вход каскадной схемы, в процессе рецикла.
-	характерным недостатком схемы 3 являются дополнительные затраты работы разделения по отношению к схемам 1 и 2, возникающие при обогащении разбавленного обедненным ураном отхода второго каскада схемы, загрязненного четными изотопами.
-	 Анализ эффективности предложенных каскадных схем с точки зрения потерь 235U показал, что перспективными вариантами для дальнейшей технико-экономической проработки являются каскадные схемы 1 и 3.
-	 Схема 1 обеспечивает экономию чистого 235U в диапазоне ~700-850 кг в зависимости от номера рецикла, по сравнению с открытым топливным циклом на обедненном уране с концентрацией 235U – 0,13%. В результате при реализации семи последовательных рециклов в такой схеме можно сэкономить массу 235U, эквивалентную массе данного изотопа, содержащейся приблизительно в 2,94 комплектов ТВС, изготовленных из природного урана для загрузки в эквивалентный реактор. При этом схема 1 на каждом рецикле позволяет извлечь более 80% от массы 235U из исходного регенерированного урана, поступившего на обогащение. Отметим, что основная доля (более 90% или ~4 т) потерь изотопа 235U в схеме 1 на каждом рецикле обусловлена потерями данного изотопа в отвале каскада 3, который нарабатывает НОУ-разбавитель из обедненного урана. Однако данные потери можно уменьшить при изменение отвальной концентрации 235U в каскаде 3 на оптимальное с точки зрения экономических затрат значение. Кроме того, эти потери не связаны с решением основной задачи – минимизацией потерь изотопа 235U из регенерированного уранового сырья. 
-	Несмотря на видимое отсутствие потерь 235U в загрязненной фракции схемы 2, в данной схеме, ввиду искусственного повышения содержания четных в получаемом продукте с каждой последующей перегрузкой возрастает масса отхода (P2) и, соответственно, масса направляемого в него изотопа 235U. Тем самым эффект от возврата изотопа 235U в цикл нивелируется его потерями вследствие увеличения потока загрязненной фракции, которое происходит из-за роста концентраций четных изотопов в исходной смеси. В результате схема 2 не обеспечивает преимуществ в экономии 235U в процессе рециклирования, по отношению к схемам 1 и 3.
-	Схема 3 обеспечивает экономию чистого 235U в диапазоне ~550-700 кг в зависимости от номера рецикла, по сравнению с открытым топливным циклом на обедненном уране с концентрацией 235U – 0,13%. Таким образом, по этому показателю схема 3 оказывается немного менее эффективной, чем схема 1, но более эффективной, чем схема 2.
Оценки показали, что при реализации семи последовательных рециклов в такой схеме можно сэкономить массу 235U, эквивалентную массе данного изотопа, содержащейся приблизительно в 2,7 комплектах ТВС, изготовленных из природного урана для загрузки в эквивалентный реактор. Отметим, что основная доля потерь изотопа 235U в схеме 3 на каждом рецикле обусловлена потерями данного изотопа в отвалах каскадов 3 и 4. Частично данные потери можно уменьшить при изменении отвальной концентрации 235U в каскаде 3 на оптимальное с точки зрения экономических затрат значение. 
Таким образом, с точки зрения эффективности вовлечения 235U в топливный цикл схема 3 несколько уступает схеме 1. Достоинством схемы 3 можно считать отсутствие нештатных отходов в виде загрязненной четными изотопами фракции.
3.	Для определения эффективности предложенных каскадных схем 1 и 3 проведено сравнение их интегральных характеристик с ординарным каскадом в условиях открытого топливного цикла на обедненном уране и природном уране, а также с доступными данными для метода обогащения регенерированного урана на базе АО «ПО «ЭХЗ» в условиях топливного цикла с использованием природного урана. Результаты проведенного сравнения показали, что предлагаемые решения (схемы 1 и 3) обеспечивают преимущества в интегральных характеристиках (затраты работы разделения и расход сырья), как по сравнению ординарным каскадом, так и по сравнению со схемой АО «ПО «ЭХЗ». Например, схема 1 позволяет при затратах работы разделения, практически эквивалентных случаю открытого ЯТЦ на природном уране, обеспечить экономию природного урана на уровне 16%, что соответствует схеме АО «ПО «ЭХЗ», у которой, однако, затраты работы разделения по отношению к открытому ЯТЦ больше на 10%. 
Схема 3 при приблизительно равных со схемой АО «ПО «ЭХЗ» показателях по работе разделения (на 10% выше, чем для открытого ЯТЦ) обеспечивает более высокую экономию природного урана, составляющую (19% против 16% у схемы АО «ПО «ЭХЗ»).
В случае реализации ЯТЦ только с использованием обедненного урана с обогащением не выше 0,13% во всех вариантах происходит существенное (в несколько раз) увеличение затрат работы разделения. Однако схемы 1 и 3 позволяют решить задачу, имея преимущества в величине работы разделения 12 и 10%, соответственно. Важно подчеркнуть, что в случае использования только обедненного урана указанные величины преимущества по работе разделения соответствуют примерно 1000 млн. ЕРР при изготовлении 252 т товарного НОУ (по металлу).
Таким образом, предлагаемые схемы оказываются более эффективными с точки зрения комбинации таких интегральных характеристик, как затраты работы разделения и расход природного урана (в случае реализации топливного цикла с природным ураном).
Полученные оценки интегральных характеристик каскадных схем для топливного цикла с использованием только обедненного урана свидетельствуют о целесообразности оценки возможности реализации ЯТЦ при таких условиях с учетом масштабов доступных производственных мощностей по обогащению урана.
4.	Для выбора конкретного варианта каскадной схемы с целью дальнейшей практической реализации необходим детальный технико-экономический анализ каждой из схем на основе их интегральных показателей (расходные характеристики, затраты работы разделения и пр.) в контексте всей цепочки стадий ЯТЦ и с учетом возникающих в этой цепочке изменений при использовании регенерата урана по отношению к открытому топливному циклу. 
Помимо этого, необходима проработка технологических проблем каждой из схем, в частности, с точки зрения возможности эксплуатации и обслуживания оборудования в условиях работы с материалами, имеющими более высокую, чем природный уран удельную активность. Например, подобные условия возникают в «очистительных» каскадах, выделяющих в легкую фракции α-активные изотопы 232U и 234U. 



Использование уранового регенерата для производства топлива легководных энергетических реакторов позволит: 
\begin{enumerate}
  \item сократить объем захоронения радиоактивных отходов; 
  \item обеспечить экономию природного урана;
  \item сэкономить затраты работы разделения (по сравнению со случаем обогащения природного урана) при дообогащении данного материала в разделительном каскаде. 
\end{enumerate}

Таким образом, вовлечение урановой составляющей отработанного топлива в ядерный топливный цикл реакторов, составляющих основную долю парка энергоблоков, позволит увеличить рентабельность электрогенерации на АЭС.

% В частности, Росатом, внедряя описанные в работе технологии, уже сегодня формирует более конкурентоспособные коммерческие предложения в части топливных поставок, а также организует эффективный переходный период на двухкомпонентную структуру ядерной энергетики. Такой подход позволяет ресурсоэффективнее воплощать глобальную стратегию замыкания ЯТЦ, осуществляя рецикл топлива в том числе с помощью имеющихся реакторных мощностей парка ВВЭР.


% \begin{enumerate}
%   \item На основе анализа \ldots
%   \item Численные исследования показали, что \ldots
%   \item Математическое моделирование показало \ldots
%   \item Для выполнения поставленных задач был создан \ldots
% \end{enumerate}
