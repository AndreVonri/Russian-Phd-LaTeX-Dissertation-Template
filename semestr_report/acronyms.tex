\chapter*{Список сокращений и условных обозначений} % Заголовок
\addcontentsline{toc}{chapter}{Список сокращений и условных обозначений}  % Добавляем его в оглавление
\noindent
%\begin{longtabu} to \dimexpr \textwidth-5\tabcolsep {r X}
\begin{longtabu} to \textwidth {r X}
% Жирное начертание для математических символов может иметь
% дополнительный смысл, поэтому они приводятся как в тексте
% диссертации

\(q_0\) & коэффициент разделения\\
\(N\) & длина каскада (число ступеней)\\

\(\begin{rcases}
f\\
N+1-f
\end{rcases}\)  &
число ступеней в обеднительной и обогатительной частях.
\\

\(\begin{rcases}
    n\\
    k
    \end{rcases}\)  &
    индексы целевого ($^{235}$U) и опорного компонент.
\\

\(\begin{rcases}
    F_i\\
    P_i\\
    W_i
    \end{rcases}\)  &
    потоки питания, отбора и отвала, где \textit{i} -- индекс каскада.
\\

\textbf{НОУ} & низкообогащенный уран \\
\textbf{ВОУ} & высокообогащенный уран\\
%\textbf{RepU} & регенерированный уран \\
\textbf{ОГФУ} & обедненный уран\\
%\textbf{Toxic Waste} & смесь с высоким содержанием минорных изотопов, классифицируемая как токсичный отход, требующий дорогостоящего хранения\\
%\textbf{mix} & смешение изотопных композиций\\


\end{longtabu}
\addtocounter{table}{-1}% Нужно откатить на единицу счетчик номеров таблиц, так как предыдущая таблица сделана для удобства представления информации по ГОСТ
