\chapter{Результаты исследований}\label{ch:ch1}

\section{Ядерное нераспространение}\label{sec:ch1/sec1}
\subsection{Анализ сценариев переключения}\label{sec:ch1/sec1}

Рассмотрен модельный сценарий переключения ядерного материала из топлива легководного реактора типа ВВЭР-1000 или ВВЭР-1200,  изготовленного из регенерированного урана. Предполагали, что АЭС находится под гарантиями МАГАТЭ. Технической целью гарантий является предотвращение или своевременное обнаружение переключения значимого количества ядерных материалов (требования своевременности обнаружения переключения значимого количества ЯМ зависят от  его категории)

\section{Сравнительный анализ известных схем}\label{sec:ch2/sec1}
\subsection{Обоснование невозможности использования ординарных схем}\label{sec:ch1/sec1}
Приведен анализ невозможности решить задачу повторного обогащения регенерата в условиях многократного рецикла. Рассмотренный состав, соответствующий пятому рециклу (пятикратному переиспользованию урана в цикле легководного реактора) был проверен на возможность получения свежего НОУ, удовлетворяющего поставленным условиям.
\paragraph{Схема с разбавлением на выходе}

\section{Разработка схем полного возврата}\label{sec:ch1/sec1}
