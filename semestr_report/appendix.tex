\chapter{}\label{app:A}

\section{Список публикаций}\label{app:A1}

\begin{enumerate}
	\item Scopus:
    \begin{enumerate}
        \item A.Yu. Smirnov, G.A. Sulaberidze, V. E. Gusev, Е.А. Andrianova, V. Yu. Blandinski, А.V. Grol, A.A. Dudnikov, V.A. Nevinitsa, P.A. Fomichenko. Applying enrichment capacities for multiple recycling of LWR uranium //  J. Phys.: Conf. Ser. 2018. V. 1099. 012001 (doi :10.1088/1742-6596/1099/1/012001)
        \item Smirnov A., Gusev V., Sulaberidze G., Nevinitsa V. A method to enrich reprocessed uranium with various initial contents of even-numbered isotopes // AIP Conference Proceedings Volume 2101, Номер статьи 020006.
        \item Gusev V.E., Smirnov A.Y., Volkov Y.N., Sulaberidze G.A., Blandinski V.Y., Grol A.V., Nevinitsa V.A. Features of light-water reactor fuel made of reprocessed uranium in terms of IAEA safeguards implementation // Journal of Physics: Conference Series
        Volume 1133, Issue 1, 2018, Номер статьи 012041
        \item Gusev V.E., Smirnov A.Yu., Sulaberidze G.A., Nevinitsa V.A. The role of Gas Centrifuge Cascades in Uranium Recycling // KERNTECHNIK (Отправлена в журнал в 2019 году (первая половина), находится на рецензировании).
    \end{enumerate}
    \item ВАК:
    \begin{enumerate}
        \item Смирнов А.Ю., Гусев В.Е., Сулаберидзе Г.А., Невиница В.А., Фомиченко П.А. ОБОГАЩЕНИЕ РЕГЕНЕРИРОВАННОГО УРАНА В ДВОЙНОМ КАСКАДЕ ГАЗОВЫХ ЦЕНТРИФУГ С ЕГО МАКСИМАЛЬНЫМ ВОЗВРАТОМ В ВОСПРОИЗВОДСТВО ТОПЛИВА // Вестник НИЯУ МИФИ. 2018. Т.7. № 6. С. 449-457
        \item В. А. Невиница, А. Ю. Смирнов, Г. А. Сулаберидзе, В. Е. Гусев, А. М. Павловичев, А. И. Щеренко, Е. В. Родионова, В. Ю. Бландинский. Топливный цикл легководного реактора с полным использованием регенерированного урана // Вестник НИЯУ МИФИ (Отправлена в редакцию 19.09.2019)
        \item Е. В. Родионова, А. Ю. Смирнов, В. А. Невиница, Г. А. Сулаберидзе, В. Е. Гусев, В. Ю. Бландинский, С. В. Цибульский. Анализ технико-экономических характеристик двойной каскадной схемы для обогащения многократно рециклированного регенерированного урана // Вопросы атомной науки и техники, сер. Физика ядерных реакторов. (Отправлена в редакцию, принята к печати)
        \item В. А. Невиница, А. Ю. Смирнов, Г. А. Сулаберидзе, В. Е. Гусев,
        А. М. Павловичев, А. И. Щеренко, Е. В. Родионова, В. Ю. Бландинский. ТОПЛИВНЫЙ ЦИКЛ ЛЕГКОВОДНОГО РЕАКТОРА С ПОЛНЫМ ИСПОЛЬЗОВАНИЕМ РЕГЕНЕРИРОВАННОГО УРАНА, ВЕСТНИК НАЦИОНАЛЬНОГО ИССЛЕДОВАТЕЛЬСКОГО ЯДЕРНОГО УНИВЕРСИТЕТА “МИФИ”, 2019, том 8, № 6,
        с. 498–506.
        \item Смирнов А.Ю., Гусев В.Е., Сулаберидзе Г.А., Невиница В.А., Фомиченко П.А. Анализ влияния ограничений по изотопам 232,234,236U в товарном НОУ на выбор способов обогащения регенерата урана в каскадах центрифуг //  Вопросы атомной науки и техники, сер. Физика ядерных реакторов. (Отправлена в редакцию в декабре 2018 года, принята к печати).
        \end{enumerate}
\end{enumerate}

\section{Участие в конференциях}~\label{app:A2}

\begin{enumerate}
    \item В качестве доклачика:
    \begin{enumerate}
        \item Конференция:  XVII International conference and School for young scholars “Physical chemical processes in atomic systems”, г. Москва, Россия. Доклад: ENRICHMENT SCHEMES FOR REPROCESSED URANIUM RECYCLING, Авторы: V.E. Gusev.
    \end{enumerate}
    \item В качестве соавтора:
    \begin{enumerate}
        \item Конференция:  15th International Workshop on Separation Phenomena in Liquids and Gases (SPLG-2019), г. Уси, Китай. Доклад: Method of Reprocessed Uranium Enrichment for NPP Fuel Supply, Авторы: A. Yu. Smirnov, V.E. Gusev, G.A.Sulaberidze
        \item Конференция:  XVII International conference and School for young scholars “Physical chemical processes in atomic systems”, г. Москва, Россия. Доклад: Physical regularities of recycling of recovered nuclear materials (uranium         and plutonium) in thermal reactor fuel, Авторы: A. Yu. Smirnov, G.A.Sulaberidze, V.E. Gusev, V.A. Nevinitsa
    \end{enumerate}
\end{enumerate}

Стажировки:
    \begin{enumerate}
        \item IV International Summer School on Engineering Computing in Nuclear Technology ECINT School 2019 (с сертификатом)
    \end{enumerate}
