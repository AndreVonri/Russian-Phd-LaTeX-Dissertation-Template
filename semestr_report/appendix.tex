\chapter{}
% \addcontentsline{toc}{chapter}{Приложение}
% \noindent

\section{Список публикаций}
\begin{enumerate}
	\item Scopus:
    \begin{enumerate}
        \item Smirnov A., Gusev V., Sulaberidze G., Nevinitsa V. A method to enrich reprocessed uranium with various initial contents of even-numbered isotopes // AIP Conference Proceedings Volume 2101. 020006, 2019. (doi :10.1063/1.5099598).
        \item Gusev V., Smirnov A., Nevinitsa V., and Volkov Yu. Proliferation resistance analysis of LWR fuel in terms of IAEA safeguards implementation // AIP Conference Proceedings Volume 2101. 020007, 2019. (doi :10.1063/1.5099599).
        \item A.Yu. Smirnov, G.A. Sulaberidze, V. E. Gusev, Е.А. Andrianova, V. Yu. Blandinski, А.V. Grol, A.A. Dudnikov, V.A. Nevinitsa, P.A. Fomichenko. Applying enrichment capacities for multiple recycling of LWR uranium //  J. Phys.: Conf. Ser. 2018. V. 1099. 012001 (doi :10.1088/1742-6596/1099/1/012001).
        \item Gusev V.E., Smirnov A.Y., Volkov Y.N., Sulaberidze G.A., Blandinski V.Y., Grol A.V., Nevinitsa V.A. Features of light-water reactor fuel made of reprocessed uranium in terms of IAEA safeguards implementation // J. Phys.: Conf. Ser. 2018. V. 1133. 012041  (doi :10.1088/1742-6596/1133/1/012041).
        \item Smirnov A., Gusev V., Sulaberidze G., Nevinitsa V. Physical and Technical Problems of Reprocessed Uranium Enrichment with Repeated Recycling in Light-Water Reactors and Ways to Solve Them //  Atomic Energy 2020 Vol. 128, No. 4, Q3 pp. 223-231 (doi :10.1007/s10512-020-00681-9). \textbf{Q3}
        \item Influence of uncertainties of isotopic composition of the reprocessed uranium on effectiveness of its enrichment in gas centrifuge cascades // J. Phys.: Conf. Ser. 781, 2017. (doi :10.1088/1742-6596/781/1/012018).
        \item   Gusev V.E. Multy-cascade enrichment schemes for reprocessed uranium recycling // J. Phys.: Conf. Ser. 2020.
        \item Analysis of the Effect of Restrictions on Isotopes 232,234,236U in Marketable LEU on the Choice of Methods for Enriching Reprocessed Uranium in Cascades of Centrifuges. Physics of Atomic Nuclei, 2021, Vol. 84, No. 8, pp. 1500–1507. A. Yu. Smirnov, V. E. Gusev, G. A. Sulaberidze, V. A. Nevinitsa, and P. A. Fomichenko.
    \end{enumerate}
    \item ВАК:
    \begin{enumerate}
        \item Смирнов А.Ю., Гусев В.Е., Сулаберидзе Г.А., Невиница В.А., Фомиченко П.А. Обогащение регенерированного урана в двойном каскаде газовых центрифуг с его максимальным возвратом в воспроизводство топлива // Вестник НИЯУ МИФИ. 2018. Т.7. № 6. С. 449-457.
        \item 	Невиница В.А., Смирнов А.Ю., СУЛАБЕРИДЗЕ Г.А., ГУСЕВ В.Е., Павловичев А.М., Щеренко А.И., Родионова Е.В.1, Бландинский В.Ю. Топливный цикл легководного реактора с полным использованием регенерированного урана // Вестник НИЯУ МИФИ. 2019. Т.8. № 6. С. 498-506.
        \item Е. В. Родионова, А. Ю. Смирнов, В. А. Невиница, Г. А. Сулаберидзе, В. Е. Гусев, В. Ю. Бландинский, С. В. Цибульский. Анализ технико-экономических характеристик двойной каскадной схемы для обогащения многократно рециклированного регенерированного урана // Вопросы атомной науки и техники, сер. Физика ядерных реакторов. 2019. № 5. С. 62-71
        \item Смирнов А.Ю., Гусев В.Е., Сулаберидзе Г.А., Невиница В.А., Фомиченко П.А. Анализ влияния ограничений по изотопам 232,234,236U в товарном НОУ на выбор способов обогащения регенерата урана в каскадах центрифуг //  Вопросы атомной науки и техники, сер. Физика ядерных реакторов.
        \end{enumerate}
\end{enumerate}

\section{Приняты к печати}
\begin{enumerate}
    \item Modeling the enrichment cascade with a loop for the utilization of subproduct of reprocessed uranium purification from even-numbered isotopes in a double cascade. NET. A.Yu. Smirnov, V.A. Nevinitsa, V.E. Gusev, G.A. Sulaberidze.
\end{enumerate}

\section{Готовятся к отправке}
\begin{enumerate}
    \item Reprocessed uranium recycling impossibility through enrichment in ordinary cascades // Nuclear Engineering and Technology \textbf{Q3}
    \item Independent involvement of light radioactive fraction from dual cascades \textbf{Q2-3}:.
\end{enumerate}

\section{Участие в конференциях}
\begin{enumerate}
    \item В качестве доклачика:
    \begin{enumerate}
        \item Конференция:  XVII International conference and School for young scholars “Physical chemical processes in atomic systems”, г. Москва, Россия. Доклад: Enrichment schemes for reprocessed uranium recycling, Авторы: V.E. Gusev.
        \item Конференция:  16th International Workshop on Separation Phenomena in Liquids and Gases (SPLG-2021), On the problems of reusing reprocessed uranium by enrichment in schemes based on ordinary cascades, Авторы: V.E. Gusev.
    \end{enumerate}
    \item В качестве соавтора:
    \begin{enumerate}
        \item Конференция:  15th International Workshop on Separation Phenomena in Liquids and Gases (SPLG-2019), г. Уси, Китай. Доклад: Method of Reprocessed Uranium Enrichment for NPP Fuel Supply, Авторы: A. Yu. Smirnov, V.E. Gusev, G.A.Sulaberidze.
        \item Конференция:  XVII International conference and School for young scholars “Physical chemical processes in atomic systems”, г. Москва, Россия. Доклад: Physical regularities of recycling of recovered nuclear materials (uranium         and plutonium) in thermal reactor fuel, Авторы: A. Yu. Smirnov, G.A.Sulaberidze, V.E. Gusev, V.A. Nevinitsa.
    \end{enumerate}
\end{enumerate}
