\chapter{Решение проблемы возврата регенерированного урана в топливный цикл}\label{ch:ch3}

\section{Теоретические основы}






Для каждой из предложенных схем разработаны оригинальные методики расчета и оптимизации ее параметров по критерию минимума суммарного потока каскадной схемы, основанная на использовании современных методов условной оптимизации функций многих переменных. С использованием разработанных методик расчета и оптимизации предложенных каскадных схем продемонстрирована возможность их использования для обогащения регенерированного урана в условиях многократного рецикла на примере взятого из литературы изотопного состава регенерата урана с повышенным содержанием четных изотопов и отвечающего пятому рециклу в топливе ВВЭР.

-	схемы 1-3 пригодны для решения задачи обогащения регенерированного урана при описанных выше условиях в рамках многократного рецикла в топливе ВВЭР. При этом каждая из схем имеет собственные достоинства и недостатки. 

-	характерным недостатком схемы 1 является наличие отхода с высоким содержанием четных изотопов (на 1-2 порядка выше, чем пределы для товарного НОУ) и 235U (до 20\% или до 90\%, в зависимости от выбранного режима работы каскадной схемы). Одним из вариантом обращения с подобным отходом может стать его перемешивание с отвалом первого каскада при обогащении регенерата. Оценки показали, что в этом случае возможно получить обедненный уран с приемлемым содержанием 232U (не выше 5·10-7\%).
-	характерными недостатком схемы 2 является возврат значительной части четных изотопов на вход каскадной схемы;
-	характерным недостатком схемы 3 являются дополнительные затраты работы разделения по отношению к схемам 1 и 2, возникающие при обогащении разбавленного обедненным ураном отхода второго каскада схемы, загрязненного четными изотопами.

-	 Анализ эффективности предложенных каскадных схем с точки зрения потерь 235U показал, что перспективными вариантами для дальнейшей технико-экономической проработки являются каскадные схемы 1 и 3.



 схема 1 на каждом рецикле позволяет извлечь более 80\% от массы 235U из исходного регенерированного урана, поступившего на обогащение


Для выбора конкретного варианта каскадной схемы с целью дальнейшей практической реализации необходим детальный технико-экономический анализ каждой из схем на основе их интегральных показателей (расходные характеристики, затраты работы разделения и пр.) в контексте всей цепочки стадий ЯТЦ и с учетом возникающих в этой цепочке изменений при использовании регенерата урана по отношению к открытому топливному циклу. 

Помимо этого, необходима проработка технологических проблем каждой из схем, в частности, с точки зрения возможности эксплуатации и обслуживания оборудования в условиях работы с материалами, имеющими более высокую, чем природный уран удельную активность. Например, подобные условия возникают в «очистительных» каскадах, выделяющих в легкую фракции α-активные изотопы 232U и 234U. 
